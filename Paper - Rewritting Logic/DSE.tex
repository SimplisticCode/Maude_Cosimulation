
\section{Synthesize of instrumentations and Parameters}\label{sc:DSE}
Using Maude's integrated LTL model checker, the Maude framework enables different kinds of design space exploration.
Design space exploration allows the co-simulation practitioner to explore how different model parameters/designs change the behavior of the system/simulation result.
The framework enables a co-simulation practitioner to experiment with different valuations of the scenario's instrumentation and parameters of the SUs.
We describe the two independently even though they can be combined.


\subsection{Instrumentation of a Scenario}
The instrumentation of the input ports is used to achieve more accurate co-simulation results as shown in \cite{Gomes2019,Oakes2021,hansen_verification_2021}.
However, finding the instrumentation that results in the most accurate simulation can be challenging since none of the existing tools to export FMUs/SUs provide this information.

We show that one can use Maude's model checker to explore the consequences of the different instrumentations of a scenario.

To explore the different instrumentations of a scenario, we define the contract of an input \texttt{noContract}, which can be used when the user does not know the input's instrumentation.
A scenario is \emph{instrumented} if none of the inputs are annotated with the contract \texttt{noContract}, otherwise the scenario is \emph{uninstrumented}.

The execution of an algorithm described in \cref{sc:synthesize} is only defined for instrumented scenarios.
% since the rewrite rules only cover inputs with \emph{delayed} or \emph{reactive} contracts.

Defining a contract to be \texttt{noContract} means that we can define a rewrite rule that non-deterministically assigns the input port to be either delayed or reactive:
\scriptsize
\begin{alltt}
rl [instr-delayed]: 
  findInstr(< SU1 : SU | inputs : (< I : Input | contract : noContract > IS) > C)
  =>
  findInstr(< SU1 : SU | inputs : (< I : Input | contract : delayed > IS) > C) .

rl [instr-reactive]: 
  findInstr(< SU1 : SU | inputs : (< I : Input | contract :   noContract > IS) > C)
  =>
  findInstr(< SU1 : SU | inputs : (< I : Input | contract : reactive > IS) > C) .

crl [remove-findInstr]: findInstr(CONF) => CONF 
  if scenarioInstrumented(CONF) .
\end{alltt}
\normalsize
The rule applies if the scenario has been wrapped inside the constructor \texttt{findInstr}.
The rule find an input \texttt{I} of an SU \texttt{SU1} with the contract \texttt{noContract} and makes the input either \texttt{delayed} or \texttt{reactive}.
Maude keeps instrumenting the scenario until the scenario is instrumented and satisfies the equational predicate \texttt{scenarioInstrumented}, where the rule \texttt{remove-findInstr} .

The different instrumentation of an uninstrumented scenario can be explored using the rule:

\small
\begin{alltt}
crl [findInstrumentation]: findContacts(INIT) => CONF
    if findInstr(INIT) => CONF
    \(\land\) empty == tarjan(CONF) *** No loops
    \(\land\) runAnyAlgorithm CONF => run: ORC on: FINAL with: SIMDATA
    \(\land\) shouldSatisfy(FINAL) .
\end{alltt}
\normalsize
The rule generates all the different instrumentations of the scenario that satisfy specific properties.
It does so by first instrumenting the scenario to generate the instrumented scenario ``CONF''.
Hereafter it simulates the instrumented scenario ``CONF'' to check the resulting simulation.

The user can limit the explored instrumentations in desired ways by defining specific properties of the instrumentation.
For example, in the rule above, it is specified that none of the instrumentations must generate a complex scenario with algebraic loops.

The co-simulation practitioner can also specify concrete constraints on the simulation result.
The constraints are above specified using the predicate \emph{shouldSatisfy}.

The different instrumentations are explored using Maude's search command.

\begin{alltt}
  \small
  search findContacts(Scenario) =>! C:Configuration .
\end{alltt}
  
This returns all the configurations/instrumentations of the scenarios that satisfy the desired properties defined in the above rule.

\subsubsection{Parameters of the Simulation Units}
An SU may have different parameters affecting its behavior; an example is the desired temperature of an incubator.
This means that different parameters potentially lead to different simulation results.
Design space exploration explores how different valuations of the parameters affect the co-simulation result to find the optimal design of the system. 

The Maude framework allows design space exploration by letting the user specify the valuation a parameter as a set of possible valuations inside the \emph{choose}-operator as we shown below.
\begin{alltt}
  \small
< "tank" : SU | parameters : ("flow" |-> choose((< 1 >,< 2 >,< 30 >))) >
\end{alltt}

The \emph{choose}-operator non-deterministically picks an element of the defined set.
Using the \emph{choose}-operator and the model checker enables one to explore all the combinations of different valuations for the parameters.
The combinations are explored using the following rewriting rule:

\begin{alltt}
  \small
  crl [dse] : selectParams(UNITIALIZEDCONF) => CONF 
  if UNITIALIZEDCONF => CONF
  \(\land\) runAnyAlgorithm CONF => 
      run: < ALG : AlgData | Initialization : emptyList, 
      CosimStep : emptyList, Termination : emptyList > 
      on: FINALSTATE
      with: SIMULATIONDATA
  \(\land\) above10(FINALSTATE) .
\end{alltt}

The rule selects a parameter valuation and uses the valuation to perform a simulation.
After a simulation, we evaluate the design by checking the simulation's final state to ensure it satisfies specific user-defined properties.
In the above rule, the valuation must ensure that the simulation satisfies the predicate \emph{above10}.

The search of the design space exploration is in Maude performed using the search command starting from a scenario with multiple valuations.
\begin{alltt}
  \small
  search selectParams(ScenarioDSE)  =>! C:Configuration .
\end{alltt}

\subsubsection{Combining Design Space Exploration of the Instrumentation and Parameters}
The two searches described above can be combined. A co-simulation practitioner can simultaneously experiment with the instrumentation and the parameters to obtain the most accurate simulation while searching for the optimal system design.

This is concretely done by defining an uninstrumented scenario with changeable parameters.
% \subsection{Limitation of the approach}
% We have performed the analyses on different scenarios. 
% The state-space explosion is, of course, also applying to our work. 
% However, we have been able to find the correct instrumentation and the contracts on systems with \simon{Create big scenario and test} SUs.

% Also, we do not currently support adaptive co-simulations~\cite{Inci2021} where the instrumentation changes during the simulation. 
% However, this is our impression that this feature trivial can be added.