
\section{Synthesizing Instrumentations and SU Parameters}\label{sc:DSE}
Our framework enables different kinds of design space exploration to allows the co-simulation practitioner to explore how different model parameters/designs change the behavior of the system.
The co-simulation practitioner can experiment with different instrumentations and values of the SU parameters.

\subsection{Instrumentation of a Scenario}
Finding a good instrumentation of the input ports (i.e. deciding whether an input port should be \texttt{reactive} or \texttt{delayed}) is important not only to achieve accurate co-simulation results as shown in \cite{Gomes2019,Oakes2021,hansen_verification_2021}, but also because some instrumentations of a scenario can lead to algebraic loop where other instrumentations do not.

However, finding the instrumentation that results in the most accurate simulation can be challenging.% since none of the existing tools to export FMUs/SUs provide this information.

We use reachability analysis to explore the consequences of different instrumentations of a scenario to find the instrumentation that yields the most accurate simulation.
To explore different instrumentations of a scenario, we create a \emph{partially instrumented} scenario, where the some of the inputs have the contract \texttt{noContract}, which means they are not \texttt{reactive} or \texttt{delayed}.
%A scenario is \emph{instrumented} if none its inputs are declared with the contract \texttt{noContract}, otherwise the scenario is partially instrumented.
\begin{example}
  The following water tank scenario is partially instrumented:
\scriptsize
\begin{alltt}
eq simulationUnitsNotInstrumented = 
< "tank" : SU | parameters : ("flow" |-> <\,100\,>),  localState : ("waterlevel" |-> <\,0\,>),
                time : 0,  fmistate : Instantiated, canReject : false, 
                inputs : (< "valveState" : Input | value : <\,0\,>, type : integer, time : 0,
                                                   contract : \emph{noContract}, status : Undef >), 
                outputs : (< "waterlevel" : Output | value : <\,0\,>, type : integer, time : 0,
                                                     status : Undef, dependsOn : empty >) >

< "ctrl" : SU | parameters : (("high" |-> <\,5\,>) , ("low" |-> <\,0\,>)), canReject : false, 
                localState : ("valve" |-> <\,false\,>), fmistate : Instantiated, time : 0, 
                inputs : (< "waterlevel" : Input | value : <\,0\,>, type : integer, time : 0,
                                                   contract : \emph{noContract}, status : Undef >), 
                outputs : (< "valveState" : Output | value : <\,0\,>, type : integer, time : 0,
                                                     status : Undef, dependsOn : empty >) > . 

\end{alltt}
\normalsize

\end{example}

A partially instrumented scenario has the form \texttt{findInstr}(\emph{scenario}) and becomes an ordinary scenario when all ports have been instrumented (rule \texttt{remove-findInstr}).
The rules \texttt{instr-delayed} and \texttt{instr-reactive} set uninstrumented input ports to be either \texttt{delayed} or \texttt{reactive}:

\scriptsize
\begin{alltt}
rl [instr-delayed]: 
  findInstr(< SU1 : SU | inputs : (< I : Input | contract : \emph{noContract} > IS) > C)
  => findInstr(< SU1 : SU | inputs : (< I : Input | contract : \emph{delayed}  > IS) > C) .

rl [instr-reactive]: 
  findInstr(< SU1 : SU | inputs : (< I : Input | contract :   \emph{noContract}  > IS) > C)
  => findInstr(< SU1 : SU | inputs : (< I : Input | contract : \emph{reactive}  > IS) > C) .

crl [remove-findInstr]: findInstr(CONF) => CONF if scenarioInstrumented(CONF) .
\end{alltt}
\normalsize

The different instrumentations of a partially instrumented scenario are found and explored using the following rule:

\small
\begin{alltt}
crl [findInstrumentation]: findContracts(INIT) => CONF
    if findInstr(INIT) => CONF
    /\char92 empty == tarjan(CONF)            *** \emph{no algebraic loops}
    /\char92 runAnyAlgorithm CONF => run: ORC on: FINAL with: SIMDATA
    /\char92 simulationDone(ORC)
    /\char92 shouldSatisfy(FINAL) .
  \end{alltt}
\normalsize

\noindent The rule \texttt{findInstrumentation} generates an instrumented scenario \texttt{CONF} from the partially instrumented scenario \texttt{INIT}.
The instrumented scenario \texttt{CONF} is simulated using the rule \texttt{runAnyAlgorithm} to explore its simulation result.
The explored instrumentations can be restricted as desired by defining specific properties the instrumented version \texttt{CONF} of the partially instrumented scenario \texttt{INIT} must satisfy.
For example, in the rule above, the condition \texttt{empty == tarjan(CONF)} specifies that none of the instrumentations must introduce algebraic loops in the scenario.

The instrumentations can be selected based on the simulation they produce. 
For example, we have stated that the simulation above must satisfy the condition \texttt{shouldSatisfy} that says the water level of the tank in the final state \texttt{FINAL} should be in a specific range.

\begin{example}
  The following Maude command finds all instrumentations of the partially instrumented scenario that satisfies criteria above: 

\small
\begin{alltt}
search findContracts(simulationUnitsNotInstrumented) =>! C:Configuration .
\end{alltt}
\normalsize

The search returns the following three instrumentations (with parts replaced by `\texttt{...}'):

\scriptsize
\begin{alltt}
Solution 1
C:Configuration --> `...'
< "ctrl" : SU | inputs : < "waterlevel" : Input | contract : delayed > `...' >
< "tank" : SU | inputs : < "valveState" : Input | contract : reactive > `...' >

Solution 2
C:Configuration --> `...'
< "ctrl" : SU | inputs : < "waterlevel" : Input | contract : reactive > `...' >
< "tank" : SU | inputs : < "valveState" : Input | contract : delayed > `...' >

Solution 3
C:Configuration --> `...'
< "ctrl" : SU | inputs : < "waterlevel" : Input | contract : delayed > `...' >
< "tank" : SU | inputs : < "valveState" : Input | contract : delayed > `...' >

\end{alltt}
\normalsize

\end{example}

\subsection{Synthesizing SU Parameters}
An SU may have different parameters.
%This means that different parameters potentially lead to different simulation results.
Design space exploration explores how different values of the parameters affect the co-simulation result in order to find the optimal design of the system. 

Our Maude framework allows design space exploration by letting the user specify a finite set possible values of a parameter using the \texttt{choose} operator, non deterministically selects an element in the set using a rule \texttt{choose(v, vs) => v}.

\begin{example}\label{ex:dse}
  We want find the parameter \texttt{flow} of the water tank example such that the final state satisfies the predicate below:

  \small
  \begin{alltt}
op above10 : Configuration -> Bool .
eq above10(CONF\,< "tank" : SU\,|\,localState\,:\,"waterlevel" |-> <\,V\,> >) = V\,>\,10\,.  
  \end{alltt}
  \normalsize

  \noindent This predicate says that the water level should be higher than 10.
We would like to select a value for the parameter \texttt{flow} from set: 1, 2 and 30 such that the simulation satisfies the desired property:

\small
\begin{alltt}
< "tank" : SU | parameters : "flow" |-> choose(< 1 >,< 2 >,< 30 >) >
\end{alltt}
\normalsize

We use the rule below to select different parameter valuations to see if they result in a simulation that satisfies the predicate \texttt{above10}:

\small
\begin{alltt}
  crl [dse] : selectParams(UNITIALIZEDCONF) => CONF 
  if UNITIALIZEDCONF => CONF
  /\char92 runAnyAlgorithm CONF => 
           run: < ALG : AlgData | Initialization : emptyList, 
                                  CosimStep : \emph{emptyList}, 
                                  Termination : emptyList > 
           on: FINALSTATE with: SIMULATIONDATA
  /\char92 above10(FINALSTATE) .
\end{alltt}
\normalsize

%\texttt{selectParams} is a constructor to ensure that no other rewrite rules can rewrite \texttt{UNITIALIZEDCONF}.
\noindent The different valuations are explored using the command:

\small
\begin{alltt}
  search selectParams(ScenarioDSE)  =>! C:Configuration .
\end{alltt}
\normalsize

\texttt{ScenarioDSE} is our water tank scenario where the parameter \texttt{flow} is defined using the \texttt{choose}-operator as shown above.
The search returns all valuations whose simulations lead to a final state that satisfy the predicate \texttt{above10}.
\small
\begin{alltt}
C:Configuration --> `...'
  < "tank" : SU | parameters : "flow" |-> < 30 > >
\end{alltt}
\normalsize

\end{example}

By having \texttt{noContract} ports and \texttt{choose}(\emph{...}) we can also \emph{simultaneously} synthesize desired instrumentations and parameter values.

% \subsection{Combining Design Space Exploration of the Instrumentation and Parameters}
% We can combine the two searches described above to simultaneously experiment with the instrumentation and the parameters of a scenario.

% \begin{example}
%   We can explore different instrumentations and parameters by defining a partially instrumented scenario with configurable parameters:

% \scriptsize
% \begin{alltt}
% eq simulationDSE =  
% < "tank" : SU | parameters : ("flow" |-> choose((< 1 >,< 2 >,< 30 >))), 
% localState : ( "waterlevel" |-> < 0 > ) , time : 0, 
% inputs : (< "valveState" : Input | value : < 0 >, type : integer, time : 0, 
%                                    contract : noContract, status : Undef  >), 
% outputs : (< "waterlevel" : Output | value : < 0 >, type : integer, time : 0, 
%                                      status : Undef, dependsOn : empty >), 
% fmistate : Instantiated, canReject : false >

% < "ctrl" : SU | parameters : (("high" |-> < 20 >) , ("low" |-> < 0 >)), 
% localState : ( "valve" |-> < false >), time : 0, 
% inputs : (< "waterlevel" : Input | value : < 0 >, type : integer, time : 0, 
%                                    contract : noContract, status : Undef  >), 
% outputs : (< "valveState" : Output | value : < 0 >, type : integer, time : 0, 
%                                      status : Undef, dependsOn : empty >), 
% fmistate : Instantiated, canReject : false > .
% \end{alltt}

% The parameters and instrumentations can now be explored using the search command 
% \normalsize

% The scenario can be explored using the search command from \cref{ex:dse}. 
% %The search finds that the following instrumentations and parameters satisfy 
% \end{example}


% \subsection{Limitation of the approach}
% We have performed the analyses on different scenarios. 
% The state-space explosion is, of course, also applying to our work. 
% However, we have been able to find the correct instrumentation and the contracts on systems with \simon{Create big scenario and test} SUs.

% Also, we do not currently support adaptive co-simulations~\cite{Inci2021} where the instrumentation changes during the simulation. 
% However, this is our impression that this feature trivial can be added.