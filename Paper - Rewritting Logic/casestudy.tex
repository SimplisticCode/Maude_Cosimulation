\section{Analysis techniques}\label{sc:casestudy}

The formalization in Maude enables several analyses of both the system of interest and the orchestration algorithm itself. 
This section walks through the different analyses types.

\subsection{Confluent Algorithms}
The described techniques for synthesizing orchestration algorithms will not always lead to one unique orchestration algorithm.
In fact, the rules often permit the synthesizing of multiple distinct orchestration algorithms that all respects the contracts of the scenario.

However, using Maude we can show an important property of the set of allowed - mainly that all of them leads to the same co-simulation result.

This analysis is carried out by Maude's built in LTL model checker using the following command:
$Add LTL expression$
This checks that all possible allowed rewriting of the systems end to same unique final state.

\subsection{Checking Safety properties}
The formalization enables a designer of a CPS to define their own $\fdoStep{}$-functions for the different SUs.  

This allow early and iterative verification of a prototype of the system. 
The perhaps most interesting verification is checking safety properties of the system using Maude's built in LTL-model checker. 

\simon{Should we describe possible system and analysis here?}
The Water tank example to show how model checking can be useful to do exploratory analysis of the system.

\subsection{Extracting Algorithms}
Maude's notation of external objects \simon{Add reference to the Maude manual} have been used to connect the developed tool with other tools. 
This means that different orchestration engines can connect to the tool to use its capabilities to generate and synthesize an implementation aware orchestration algorithm of a given scenario.

\subsection{Running co-simulations in Maude}
The notion of external objects can also be used to connect the Maude model with real world SUs.
This ability has several interesting applications both in the scope of a single SU and a scenario.

Connecting Maude with a single SU allows one to carry out extensive test using 

This to use Maude directly for running the co-simulation.
