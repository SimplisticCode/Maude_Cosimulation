\section{Analysis techniques}\label{sc:casestudy}
The formalization in Maude enables several analyses of both the system of interest and the orchestration algorithm itself. 
This section describes the different analyses.

\subsection{Confluent Algorithms}
The technique to synthesize orchestration algorithms of a given scenario will not always lead to one unique orchestration algorithm.
In fact, we can often synthesize multiple distinct orchestration algorithms that all respect the scenario's contracts.

However, we can establish that all the different algorithms are confluent and lead to the same co-simulation result using Maude.

This analysis is carried out by Maude's built-in LTL model checker using the following command:
\begin{align}
    $Add LTL expression$
\end{align}

The query checks that all possible allowed rewriting of the systems lead to the same final state or co-simulation result.
This ensures that a generated algorithm will lead to the  

\subsection{Checking Safety properties}
The formalization in Maude enables a designer of a CPS to define their own $\fdoStep{}$-functions for the different SUs.  

This allows early and iterative verification of a prototype of the system. 
We can use this approach 
The perhaps most exciting verification is checking the safety properties of the system using Maude's built-in LTL-model checker. 

\simon{Should we describe possible systems and analysis here?}
The Water tank example to show how model checking can be useful to do exploratory analysis of the system.
