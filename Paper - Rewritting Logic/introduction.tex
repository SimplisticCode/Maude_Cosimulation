% % Motivation for co-simulation
% Cyber-physical systems (CPSs) are omnipresent and part of the critical infrastructure.
% A CPS is a hybrid system that embodies physical processes controlled
% by digital devices. Examples include everything from a controller
% opening and closing a valve to regulate the water level in a tank to
% autonomous vehicles. 
% CPSs are becoming increasingly complex and critical~\cite{4519604},
% which leads to the desire for techniques to assist in their
% development. 
% Traditional modeling and simulation techniques, where a solver
% algorithm describes a model, are no longer sufficient to cope with
% the integrated development processes of such systems~\cite{FMI2014},
% which consists of heterogeneous subsystems typically developed using
% different tools and formalisms  
%
% % Why and what is co-simulation
% %Co-simulation is the study of how to coordinate multiple black-box
% simulation units (SUs) developed using different tools and
% formalisms. 
% Co-simulation solves the shortcomings of the traditional modeling
% techniques by letting the user explore the behavior of a system
% described as a composition of black-box subsystems
% (SUs)~\cite{Kubler2000,Gomes2018}. 
%
% An SU implements a well-defined interface and represents a subsystem
% by calculating its behavioral trace using a dedicated solver. 
% An example of such an SU is a Functional Mock-up Unit (FMU) defined
% by the Functional Mock-up Interface Standard~\cite{FMI2014} (FMI),
% which inspires the notion of an SU in this paper.  
% FMI is widely adopted, supported by many tools~\cite{Tools_FMI}, and
% used commercially. 
%
% %How to test and verify an OA - need for correct by construction algorithm
% The overarching challenge of co-simulation is ensuring accurate
% simulation results due to the many potential error
% sources~\cite{Gomes2018}. 
% The simulation result depends not only on the correctness of the
% individual SUs but also on the co-simulation algorithm that composes
% them, which is often
% overlooked~\cite{Gomes2019,Gomes2018f,Schweizer2015c}. 
% The co-simulation algorithm must be specifically tailored to the
% instrumentation of the scenario to ensure an accurate simulation
% result~\cite{hansen_verification_2021,Oakes2021}. 
% Not considering the scenario's instrumentation can lead to
% inaccurate co-simulation results, which can be extremely difficult
% to debug and fix in practice. 
% The challenge has partly been addressed by the papers
% \cite{Gomes2019,Oakes2021,thrane2021} that show how to synthesize
% tailored co-simulation algorithms. 
% Nevertheless, the approaches do not consider scenarios where the
% instrumentation is unknown or the SUs are parametric. 
%
% %Design space evaluation
% Co-simulation creates an environment where SUs can be composed to
% explore and analyze different system's design to find the optimal
% design~\cite{dse,gamble_design_2014}; the process is called design
% space exploration. 
%
% We propose a framework in Maude where a co-simulation practitioner
% can synthesize  and execute tailored co-simulation algorithms while
% exploring  different system designs for both simple and complex scenarios.
% Complex scenarios are subject to algebraic loops denoting cyclic dependencies between the SUs or step rejections, where an SU refuses specific future state evaluations because it implements error estimation.
% Complex scenarios are more challenging to simulate since the co-simulation algorithm needs to adapt to the behavior of the SUs~\cite{thrane2021} to solve algebraic loops and ensure that all SUs move in lockstep.
%
% Our framework enables a co-simulation practitioner to explore the consequences of changing the scenario's instrumentation and SU parameters. 
% Co-simulation practitioners can thereby use their domain knowledge to place constraints on the co-simulation result to let Maude find the parameters and instrumentation that result in a co-simulation with desirable properties. 
%
%
% =============================

\section{Introduction}

Modern  cyber-physical systems (CPSs),  such as,  e.g.,  nuclear power
plants, cars, and airplanes, consist of multiple
heterogeneous subsystems that    are developed by different
companies using different tools and formalisms~\cite{Paris19}.
These companies will usually not share their models for commercial reasons.
There is nevertheless  a need to determine   how
these subsystems 
interact and to  explore and analyze different design choices as early
as possible~\cite{4519604}. 
One way of addressing this need is to use, for each subsystem,  an
interface that provides an abstraction of that subsystem. 
\emph{Simulation
units} (SUs) provide such abstractions and are widely used in
industry. A class  of SUs are   described by
the Functional  Mock-up Interface Standard~\cite{FMI2014} (FMI), which
is used commercially and is  supported by many
tools~\cite{Tools_FMI}. 
An SU implements a well-defined interface and represents a subsystem
by computing   its behavioral trace using a dedicated solver.

\emph{Co-simulation}~\cite{Kubler2000,Gomes2018} addresses the need to
simulate a  CPS given as the
composition of  such black-box SUs.
Co-simulation transforms a continuous system to a discrete simulation
with discrete  interactions between the different SUs.
Furthermore, a \emph{digital twin} can be a co-simulation connected to
a  physical  systems.

The   objective of a
co-simulation is to capture as accurately as possible   
the behavior of  the modeled system. This is challenging,  due
to discretization,  cyclic dependencies between the SUs,  and the fact
that very few  assumptions be made about the SUs: an SU may, e.g., be
unable to predict  its future state at the next desired point in time.
A \emph{co-simulation algorithm} is responsible for orchestrating the
interaction of the SUs:   it determines how and when the different SUs
interact. 

Since the co-simulation algorithm should make   the virtual
system correspond  to its physical
counterpart,   the virtual system can be analyzed,
and different design  choices can be explored,  to accurately predict
the behavior of the  system to be built. 
%  
However, the FMI standard is only informally described, and has been
shown to be inconsistent \cite{sampaio_behavioural_2016}.  For both of
the above reasons,  there is
 a need for 
formal methods to provide a formal semantics for co-simulation and to
provide early model-based formal analysis of the co-simulations.   

However, providing a formal semantics to co-simulation is challenging,
due to, e.g.,  the
complex behavior of  the SUs, and the need to resolve cyclic
dependencies between 
the SUs by fixed-point computations and to  perform step negotiation
to ensure that all SUs move in  lockstep.
Rewriting logic~\cite{Mes92}, with its  modeling language and
high-performance  analysis tool Maude~\cite{maude-book}, 
 should be a suitable  formal method for co-simulation: 
 Its expressiveness allows us to conveniently specify both  
 complex dynamic behaviors and 
sophisticated  functions  (e.g., for detecting and resolving cyclic
dependencies), and 
Maude  provides  automatic  formal analysis capabilities for
correctness analysis and 
design space  exploration.
Maude also supports connections to \emph{external objects},
which means  that  Maude   should be able to 
 orchestrate  the composition of real  external components. 

In this paper we present a formal framework for representing
co-simulation in Maude. 
We give a formal model for co-simulation beyond the FMI 2.0 standard, 
also covering  feed-through constraints, input instrumentations, and
step rejection.  
We then use Maude to synthesize and execute suitable
(scenario-specific) 
co-simulation  algorithms, which enables the formal analysis of the
resulting  co-simulation. 
We also show how Maude can be used to synthesize
instrumentations,  parameter values,  and co-simulation algorithms for
such complex  scenarios so that the resulting system satisfies
desired properties. 
% 
As discussed in \Cref{sc:related}, to the best of our
knowledge this paper presents  the first formal
framework that covers  design space exploration of complex
co-simulation scenarios  with algebraic loops and step
rejection,  and that also  synthesizes correct-by-construction
co-simulation algorithms and  parameters for  such  scenarios.

From a Maude perspective, we found that using rewrite conditions in
 rules  allowed us to  easily and
elegantly  solve quite   challenging problems in co-simulation.


The rest of the paper is structured as follows.
\Cref{sc:background} provides  necessary background to Maude and
co-simulation. 
\Cref{sec:model} presents a Maude model of co-simulation
scenarios and SU  behaviors.
\Cref{sc:synthesize} shows how correct-by-construction co-simulation
algorithms can be  synthesized and executed in Maude.
\Cref{sc:DSE} describes how to synthesize instrumentation and
parameter values such  that resulting co-simulation satisfies desired
properties. 
\Cref{sc:related} discusses related work and \Cref{sc:summary} gives some
concluding  remarks.