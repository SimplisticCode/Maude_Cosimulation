\section{Introduction}\label{sc:introduction}
\begin{itemize}
    \item Motivation for co-simulation
    \begin{itemize}
      \item Industry 4.0
      \item Development of CPS
      \item The result of co-simulation
      \begin{itemize}
      \item How to test and verify an OA - need for correct by construction algorithm
      \item Design Space Exploration of the system - model checking 
      \item Testing the configurations of the systems - model checking the contracts.
      \item Dynamic observation of the system
      \item Satisfying the contracts
      \item Checking safety properties (possible to reach a bad state). - model checking might be needed because although each SU might have deterministic behavior, we might not know this - we have to consider a range of possible behaviors.
      \item General problem with writing sound SU.
      \item Time 
    \end{itemize}
    \end{itemize}
    \item Background
    \begin{itemize}
      \item Rewriting Logic
      \item Co-simulation
    \end{itemize}
    \item The Maude model
    \item Analyses
    \begin{itemize}
      \item Confluent and deterministic algorithm
      \item Verification of Algorithms
      \item Design Space Exploration
      \item Step negotiation (Non-deterministic behavior)
      \item Contracts
      \item Interaction with real-world SUs
    \end{itemize}
  \end{itemize}
  
  Need for formal methods:
  \begin{itemize}
    \item Desireable to do model checking as run-time verification during the simulation.
    \item Communication to external objects - continuous Interaction between formal tool.
    \item Step negotiation
    \item Fixed-point iteration (Algebraic loop)
    \item Complex functions - Tarjan, fixed-point iteration.
  \end{itemize}
  
  Contribution
  \begin{itemize}
    \item Address the problems - obtaining good co-simulation algorithm
    \item Perform co-simulation in Maude
    \item Step negotiation
    \item Fixed-point iteration (Algebraic loop)
    \item Complex functions - Tarjan, fixed-point iteration.
  \end{itemize}
  
  Use of Maude:
  \begin{itemize}
    \item Addresses the challenges:
    \item Non-distributed components
    \item Supports functions and is very compressive
    \item Supports model-checking - lets the user explore all different initialization of all the simulation units
    \item Model checking allows one to explore that the SU does not reach a bad state
    \item Most importantly, Maude supports communication with external objects so, in each simulation iteration, we can check and verify (model check) the algorithm used to simulate the next system step.
  \end{itemize}
  
  In this paper, we present a co-simulation Synthesizing engine and a Framework for performing various analyses in Maude of both the algorithm and the system of interest.
  \begin{itemize}
    \item Model checking approach is used to explore the system's instrumentation's design space - check all possible contracts to reach the best possible simulation result.
    \item Given a state of the co-simulation system, we synthesize all correct orchestration (one of the algorithms might actually be better than another) for the next step of the co-simulation algorithm.
    \item Model checking is used to show that all synthesized algorithms lead to a  
    \item By defining abstract models of the system in Maude, we can perform model checking to explore violations of safety properties - to 
    \item We then show have this framework can be used to run a co-simulation - to let Maude use as the orchestration engine.
  \end{itemize}
  
  We illustrate our techniques with some examples.
  
  Related work
  \begin{itemize}
    \item Not a lot of formal methods on co-simulation
    \begin{itemize}
      \item All do work on a symbolic version of the simulation units
      \item Generate only one algorithm
    \end{itemize}
  \end{itemize}

% Motivation for co-simulation
Cyber-physical systems (CPS) embody physical processes controlled by cyber elements. 
CPSs are omnipresent and control more and more critical functions in our daily life.
A CPS is typically developed in a distributed fashion using different tools and techniques. 
Such systems are becoming increasingly complex~\cite{4519604}, which leads to the desire for techniques to assist in the development of dependable CPS.
% Why and what is co-simulation
Co-simulation is one of these techniques. Co-simulation is the study of how to coordinate multiple black-box simulation units (SUs).
An SU represents a sub-system. 
The SU is responsible for computing the behavior of the sub-system, in order to compute their combined behavior, and therefore produce the global behavior of a system, as a discrete trace (see, e.g., \cite{Kubler2000,Gomes2018}).

Co-simulation allows iterative integration of constituents to explore the global system behavior without violating the constituents' intellectual property. 
The SUs are coupled by an orchestration algorithm that interacts with each SU through an interface.
An example of such an SU is a Functional Mock-up Unit (FMU) defined by the Functional Mock-up Interface Standard~\cite{FMI2014} (FMI), which inspires the notion of an SU in this paper. FMI is a widely adopted standard used commercially and supported by many tools~\cite{Tools_FMI}.

%How to test and verify an OA - need for correct by construction algorithm
The overarching challenge of co-simulation is ensuring correct and deterministic simulation results. 
Previous studies \cite{Gomes2019,Oakes2021,Gomes2018f,Schweizer2015c} show that obtaining a correct and deterministic co-simulation result requires an implementation-aware algorithm.
The algorithm should be specifically tailored to the scenario to respect the SUs' input approximation functions.

Not considering such details can lead to inaccurate co-simulation results that can be hard to debug \cite{Gomes2019,Oakes2021}, which show how contracts on the co-simulation algorithm could be constructed based on the SUs. 
However, the 

Obeying such contracts leads to a substantial reduction of co-simulation errors (see also \cref{sc:related} for more related work). 
Complex scenarios are an even more challenging class of scenarios to simulate.
These scenarios are subject to either algebraic loops or adaptive steps. 
Complex scenarios are simulated using a specific iterative algorithm \cite{thrane2021}. 
The iterative algorithm solves the algebraic loop (cyclic dependencies between the SUs) and ensures that all SUs agree on a step; the latter is referred to as step negotiation. 
Step negotiation permits the SUs to implement error estimation and refuse specific future state evaluations to minimize the simulation error while ensuring that the SUs move in lockstep.

% Checking safety properties (possible to reach a bad state). - model checking might be needed because although each SU might have deterministic behavior, we might not know this - we have to consider a range of possible behaviors.

We propose a model checking approach that allows a co-simulation practitioner to synthesize and verify co-simulation algorithms.
Furthermore, the approach enables various analyses such as design space exploration of simulation parameters and various instrumentation of the scenario to find a suitable instrumentation of the system producing the most accurate simulation.


\textit{Contribution:}
This paper describes an executable Maude formalization for synthesizing, verifying and executing co-simulation algorithms.
The approach 
The approach covers complex scenarios subject to algebraic loops and adaptive steps.

The formalization enables multiple analyses of both the generated algorithm and the system of interest. 

\textit{Structure:}
The paper starts with introducing rewriting logic, co-simulation and the verification challenge of co-simulation algorithms in \cref{sc:background}. Then, \Cref{sc:related} describes other approaches for obtaining reliable and deterministic co-simulation results. 
Finally, \cref{sc:summary} concludes.