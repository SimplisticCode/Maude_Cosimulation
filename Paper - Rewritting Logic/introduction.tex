\section{Introduction}\label{sc:introduction}
Cyber-physical systems (CPS) are omnipresent and embody physical processes being controlled by cyber elements. A CPS is typically developed in a distributed fashion using different tools and techniques. Such systems are becoming increasingly complex~\cite{4519604}, which leads to the desire for techniques to assist in the development of these.
% Why and what is co-simulation
One such technique is co-simulation: the study of how to coordinate multiple black-box simulation units (SUs), each responsible for computing the behavior of a sub-system, in order to compute their combined behavior, and therefore produce the global behavior of a system, as a discrete trace (see, e.g., \cite{Kubler2000,Gomes2018}).

Co-simulation allows iterative integration of constituents to explore the global system behavior without violating the constituents' intellectual property. 
The SUs are coupled by an orchestration algorithm that interacts with each SU through an interface.
An example of such an SU is a Functional Mock-up Unit (FMU) defined by the Functional Mock-up Interface Standard~\cite{FMI2014} (FMI), which inspires the notion of an SU in this paper. FMI is a widely adopted standard used commercially and supported by many tools~\cite{Tools_FMI}.

The overarching challenge of co-simulation is ensuring correct simulation results. Previous studies \cite{Gomes2019,Oakes2021,Gomes2018f,Schweizer2015c} have shown that obtaining a correct co-simulation result requires an algorithm specifically tailored to the scenario that respects the SUs' input approximation functions. Not considering such details can lead to hard to debug errors in the co-simulation results as highlighted in \cite{Gomes2019,Oakes2021}, where it is shown how contracts on the co-simulation algorithm could be constructed based on the SUs. Obeying such contracts leads to a substantial reduction of co-simulation errors (see also \cref{sc:related} for more related work). An even more challenging class of scenarios to simulate are complex scenarios subject to either algebraic loops or adaptive steps. Complex scenarios are simulated using a specific iterative algorithm \cite{thrane2021}. The iterative algorithm solves the algebraic loop (cyclic dependencies between the SUs) and ensures that all SUs agree on a step; the latter is referred to as step negotiation. Step negotiation permits the SUs to implement error estimation and refuse certain future state evaluations to minimize the simulation error while ensuring that the SUs move in lockstep.
We propose an approach that has been implemented as a tool. The tool lets users verify that their algorithm respects the contracts of the SUs.

\textit{Contribution:}
This paper describes an approach for verifying that a co-simulation algorithm satisfies the contracts of the scenario. The approach covers complex scenarios subject to algebraic loops and adaptive steps.
It has been implemented in UPPAAL~\cite{behrmann_uppaal_2006} and has been applied to several case studies, including an industrial case study from Boeing \cite{Gomes2019a} and complex scenarios subject to algebraic loops and step negotiation.

\textit{Structure:}
The paper starts with introducing co-simulation and the verification challenge of co-simulation algorithms in \cref{sc:background}. \Cref{sc:related} describes other approaches for obtaining reliable and deterministic co-simulation results. \Cref{sc:verification} follows with a presentation of the verification technique. \Cref{sc:casestudy} discusses a case study and \cref{sc:summary} concludes.