% Motivation for co-simulation
Cyber-physical systems (CPSs) are omnipresent and part of the critical infrastructure of the modern society.
A CPS is a hybrid system that embodies physical processes controlled by digital devices. Examples include everything from a controller opening and closing a valve to regulate the water level of a tank to autonomous vehicle.
CPSs are becoming increasingly complex and critical~\cite{4519604}, which leads to the desire for techniques to assist in their development.
Traditional modeling and simulation techniques, where a solver algorithm describes a model, are no longer sufficient to cope with the integrated development processes of such systems~\cite{FMI2014}, which consists of heterogenous subsystems typically developed using different tools and techniques~\cite{Paris19}. 

% Why and what is co-simulation
%Co-simulation is the study of how to coordinate multiple black-box simulation units (SUs) developed using different tools and formalisms.
Co-simulation solves the shortcomings of the traditional modeling techniques by letting the user couple multiple black-box subsystems (SUs) into a scenario in order to explore the behavior of the combined system as a discrete trace (see, e.g., \cite{Kubler2000,Gomes2018}).

An SU implements a well-defined interface and represents a subsystem by calculating its behavioral trace using a dedicated solver.
An example of such an SU is a Functional Mock-up Unit (FMU) defined by the Functional Mock-up Interface Standard~\cite{FMI2014} (FMI), which inspires the notion of an SU in this paper. 
FMI is widely adopted; supported by many tools~\cite{Tools_FMI} and used commercially.

%How to test and verify an OA - need for correct by construction algorithm
The overarching challenge of co-simulation is ensuring accurate simulation results due to the many potential error sources of such a simulation~\cite{Gomes2018}.
The simulation result depends not only on the SUs but also on the co-simulation algorithm that composes them in a simulation, the later often overlooked~\cite{Gomes2019,Oakes2021,Gomes2018f,Schweizer2015c}.
The co-simulation algorithm must be specifically tailored to the instrumentation of the scenario to ensure an accurate simulation result~\cite{hansen_verification_2021}.
Not considering the scenario's instrumentation can lead to inaccurate co-simulation results, which can be extremely difficult to debug and fix in practice.
The challenge has partly been addressed by the papers \cite{Gomes2019,Oakes2021,thrane2021} that show how to synthesize tailored co-simulation algorithms.
Nevertheless, the approaches do not consider scenarios with an unknown instrumentation or with configurable SU parameters.

%Design space evalution
Co-simulation does not only create an environment where SUs developed using different tools and formalisms can be composed,
The techniques also creates an environment where different system designs can be explored to find the optimal design~\cite{dse,gamble_design_2014}; the process is called design space exploration.

We propose a framework in Maude where a co-simulation practitioner can synthesize and execute tailored co-simulation algorithms while exploring different system designs for both simple and complex scenarios.
Complex scenarios are subject to algebraic loops denoting cyclic dependencies between the SUs or step rejections, where an SU refuses specific future state evaluations because it implements error estimation.
Complex scenarios are more challenging to simulate since the co-simulation algorithm needs to adapt to the behavior of the SUs~\cite{thrane2021} to solve algebraic loops and ensure that all SUs move in lockstep.

Our framework enables a co-simulation practitioner to explore the consequences of changing the scenario's instrumentation and SU parameters. 
Co-simulation practitioners can thereby use their domain knowledge to place constraints on the co-simulation result to let Maude find the parameters and instrumentation that result in a co-simulation with desirable properties. 

\paragraph{Contribution:}
% \begin{itemize}
%   \item Address the problems - obtaining good co-simulation algorithm
%   \item Perform co-simulation in Maude
%   \item Step negotiation
%   \item Fixed-point iteration (Algebraic loop)
%   \item Complex functions - Tarjan, fixed-point iteration.
% \end{itemize}
This paper describes an executable Maude framework for synthesizing and executing co-simulation algorithms for a given co-simulation potentially complex co-simulation scenario.
The framework enables synthesis of SU parameters and an instrumentation of given scenario such that the instrumented system/scenario have certain desirable properties.

\paragraph{Structure:}
\cref{sc:background} introduces the preliminaries rewriting logic, co-simulation, and design space exploration.
Then, \cref{sec:model} describes the Maude model, before \cref{sc:synthesize} describes how co-simulation algorithms are synthesized and executed in Maude.
\cref{sc:DSE} describes how our framework can be used to select the instrumentation and SU parameters of a scenario.
\cref{sc:related} describes the related work. 
Finally, \cref{sc:summary} concludes and present future work.