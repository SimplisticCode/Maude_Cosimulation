\section{Co-simulation in Maude}
This section describes how co-simulation is modeled in Maude.
We start with describing the SUs before addressing their composition in a co-simulation scenario. Finally, we describe the orchestrator that synthesizes the orchestration algorithm and runs it.
Due to space limitations, we cannot describe the entire model here. 
However, the complete model can be found online \footnote{\simon{create link}}.

\subsection{Simulation Unit}
We define an SU based on \cref{def:fmu} in Maude as the class.

\begin{alltt}
\small
class SU |
  time : Nat, 
  inputs : Configuration, 
  outputs : Configuration,
  canReject : Bool, 
  fmistate : fmiState,
  parameters : LocalState,
  localState : LocalState .
\end{alltt}

The inputs and output are also modeled as a class. 
We place the reactivity $\allreactivity$ and feedthrough $\allfeedthroughs$ as a field in these classes.

We model the interface of an SU (the functions described in \simon{Make defintions}) using equations.
These methods operate on the abstract state space described in \cref{def:runtime_state}.
An example is the $\fget{c}$ function shown below:

\begin{alltt}
  \small
  *** Updates the time and status of the outputs
  op getAction : Object OidSet -> Object .
  eq getAction(< SU1 : SU | >, empty) = < SU1 : SU | > .
  eq getAction(< SU1 : SU | time : T, 
    outputs : (< O : Output | > OS) >, (O , P)) = 
    getAction(< SU1 : SU | outputs : 
    (< O : Output | time : T, status : Def > OS) >, P) .
\end{alltt}

The function updates the 

However, it is possible to model the actual behavior of an SU by creating a custom version of the $\fdoStep{C}$ for each of the SUs. 
This can be done using both non-deterministic rewriting rules or deterministic equations.
Due to the expressiveness of Maude, it is possible to model quite complex behavior, as seen below.

\begin{alltt}
  \small
  eq step(< "tank" : SU | time : T, parameters : ("flow" |-> < FLOW >), 
                      inputs : < "valve" : Input | value : < STATE > >, 
                      outputs : < "wlevel" : Output | time : T >,
                      localState : ( "wlevel" |-> < LEVEL > ) >,
                      STEP) = 
  if STATE == 1 then 
      < "tank" : SU | time:(T+STEP), localState : ( "wlevel" |-> < 0 > ),
      outputs : < "wlevel" : Output | value : < 0 >, 
      time : (T + STEP), status : Undef > >
    else 
      < "tank" : SU | time : (T + STEP), 
      localState : ( "wlevel" |-> < LEVEL + (STEP * FLOW) > ), 
      outputs :< "wlevel" : Output|value : < LEVEL + (STEP * FLOW) >, 
      time : (T + STEP), status : Undef > > 
  fi .
\end{alltt}

The equation describes how the water level in the tank changes according to time, the inflow, and the state of the valve (open/closed).

The SUs can be composed in a co-simulation scenario.
The scenario (defined in \cref{def:cosim_scenario}) is in Maude simply a collection of SUs that are connected using a set of connections.
The reactivity and feedthrough are, as earlier described, placed directly on the ports.
We instantiate a scenario using the \emph{setup} equation that generates a \emph{GlobalState} that is defined as a scenario and orchestrator.
\begin{alltt}
  \small
  op setup : -> GlobalState .
  ceq setup = { CONF }
    if MODEL := scenario(externalConnection, simulationUnits)
    \(\land\) LOOPS := tarjan(MODEL)
    \(\land\) NeSUIDs := getSUIDsOfScenario(MODEL)
    \(\land\) CONF := calculateSNSet(MODEL OData(1,LOOPS, NeSUIDs)) .
\end{alltt}

The operator validates the scenario by checking the connections and other common mistakes.
Furthermore, it checks if the scenario is complex or not.
It does so by looking for algebraic loops and checking if step negotiation should be applied.
Algebraic loops are detected as non-trivial strongly connected components in the graph constructed by the rules described in . 
Once a scenario is constructed, it is ready to be simulated by the orchestrator, which we describe next. 

\subsection{Orchestrator}
The orchestrator is responsible for simulating the scenario.
It does so by exchanging value and progressing all of the SUs.

To do this, it keeps track of the state of the scenario, the orchestration algorithm, and the simulation data that enables it to negotiate the step duration and exchange value between the SUs.

The orchestrator is consequently divided into several classes to cover all of this.
\begin{compactitem}
  \item Simulation Data - Data used during the simulation the step duration, data to exchange, etc.
  \item Algorithm Data - The orchestration algorithm is divided into three subsequent parts. 
  \item Searching Data - Data used to solve algebraic loops and perform step negotiation.
\end{compactitem}

The orchestrator has two different functionalities.
The first is to synthesize implementation-aware orchestration algorithms tailored to the specific scenario.
The second is the execution of the orchestration algorithm. 

\subsubsection{Synthesize Algorithms}
Algorithms represented a list/sequence of the actions described in \cref{def:fmu} are synthesized using rewriting of the GlobalState object.
Rewriting logic allows one to synthesis correct-by-construction algorithms using conditional rewriting rules.
The conditional rewriting rules describe how a given action changes the abstract state of the system.
Furthermore, we can use the conditions to specify the preconditions of the given rewritings.
This is done by defining the actions described in \simon{add definitions} as conditional rewriting rules such that the state is changed according to the definitions.
Furthermore, we use the condition and the initial state to ensure that a rewriting is only possible if it satisfies its associated precondition.

\begin{alltt}
  \small
  crl [get-init] :
  < SU1 : SU | fmistate : Initialize, inputs : IS, 
    outputs : (< O : Output | status : Undef, value : V, dependsOn : FT > OS) > 
  (SU1 ! O ==> SU2 ! I)
  < OCH : SimData | values : PV > 
  < ALG : AlgoData | Initialization : ALGO >
  => < SU1 : SU | outputs : (< O : Output | status : Def > OS) > 
    (SU1 ! O ==> SU2 ! I)
    < OCH : SimData | values : insert((SU2 ! I), < 0 ; V >, PV) >  
    < ALG : AlgoData | Initialization : (ALGO ; EVENT) >
if feedthroughSatisfied(FT, IS, 0)
  \(\land\) EVENT := portEvent: Get SU: SU1 PId: O . 
\end{alltt}

The rewriting does not only change the state of the SU $SU1$, it also logs the chosen rewriting and saves the obtained value such that it later can be set on the connected input.

An orchestration algorithm is synthesized by letting Maude rewrite the system until it reaches the final simulation state where all components have reached the end time and have been unloaded.

This strategy ensures the generation of an implementation-aware algorithm because we only allow a Maude to perform an action/rewrite the system if the co-simulation state allows the rewriting to take place.

Since, we let Maude explore all possible allowed rewritings of scenario we often extract multiple implementation-aware algorithms of the same scenario.

We synthesize all the algorithms of a given scenario using the following rewrite rule.

The Synthesizer terminates and returns the constructed co-simulation algorithm once no more rewriting can take place and all SUs are unloaded.

\begin{alltt}
  \small
crl [getAlgortihm]: \{ INIT \} => \{ getOrchestrator(FINALSTATE) \}
if isInitialState(INIT)
  \(\land\) LOOPS := tarjan(INIT)
  \(\land\) SUIDsNE := getSUIDsOfScenario(INIT)
  \(\land\) SIMDATA := initialOrchestrationData(1,LOOPS,SUIDsNE)
  \(\land\) CONF := calculateSNSet(INIT) SIMDATA initialAlgorithmData(1)
  \(\land\) \{CONF\} => \{ FINALSTATE\} 
  \(\land\) allSUsinUnloaded(SUIDsNE, FINALSTATE) .
\end{alltt}


The rule constructs a GlobalState ``CONF'' from the initial state of the Scenario ``INIT''.
Hereafter, it uses the nested \simon{Is this the correct name} rewriting rule to search for all the final states ``'FINALSTATE' that can be obtained by rewriting the initial state ``CONF'' such that all SUs are correctly simulated and unloaded.

Once an algorithm has been synthesized we can execute it to to simulate a scenario.
The execution can be used to both test the scenario, but also to test and compare the different algorithms which we do in the following section.

\subsubsection{Executing Orchestraiton Algorithms}
This section describes how we run simulations using the generated algorithms.

An algorithm is executed by systematically interpreting the algorithm by going through it and executing one action at a time using rewriting.
Below we show the rule describing the $\fget{}$ action.

\begin{alltt}
crl [get] : 
    run: < ALG : AlgoData | CosimStep : ((portEvent: Get SU: SU1 PId: O) ; ALGO) >
    on: CONF
        < SU1 : SU | outputs : (< O : Output | time : T, 
        value : V, dependsOn : FT > OS), inputs : IS > 
        ( SU1 ! O ==> SU2 ! INPUT)
    with: < OCH : SimData | values : PV >
    => 
    run: < ALG : AlgoData | CosimStep : ALGO >
    on: CONF < SU1 : SU | outputs : (< O : Output | status : Def > OS) > 
                ( SU1 ! O ==> SU2 ! INPUT)
    with: < OCH : SimData | values : insert((SU2 ! INPUT), < T ; V >, PV) > 
if feedthroughSatisfied(FT, IS, T) .
\end{alltt}

We use the execution of an algorithm to perform different kids of analysis of both the algorithms and scenario itself.

We start with exploring the different implementation-aware algorithms that can be generated to simulate the same scenario.
The simulation result should not depend on the algorithm we happen to extract, therefore we would like that all implementation-aware produces the same simulation result.
We can show this by defining a rule that extracts all algorithms for a scenario and afterwards run a simulation for each of the algorithms starting from the same initial state.

\begin{alltt}
  \small
crl [runAlg] : runAnyAlgorithm INIT => run: ORC on: INIT with: SIMDATA
  if LOOPS := tarjan(INIT)
  \(\land\) SUIDsNE := getSUIDsOfScenario(INIT)
  \(\land\) SIMDATA := initialOrchestrationData(1,LOOPS,SUIDsNE)
  \(\land\) ALGO := initialAlgorithmData(1)
  \(\land\) CONF := calculateSNSet(INIT ALGO) SIMDATA 
  \(\land\) {CONF} => { FINALSTATE } 
  \(\land\) ORC := getOrchestrator(FINALSTATE)
  \(\land\) allSUsinUnloaded(SUIDsNE, FINALSTATE) .
\end{alltt}

On top of this we use Maude's search command to check if all the different allowed simulations are confluent such that they result in the same final state.

\begin{lemma}[Deterministic Orchestration Algorithms]
  All implementation-aware orchestration algorithms of a scenario produce the same co-simulation result.
\end{lemma}

This property can be verified using the following search command.

\begin{alltt}
  search (runAnyAlgorithm INIT)  =>! S:SimulationState . 
\end{alltt}
The search command uses the rewriting rules to first generate all the different orchestration algorithms for the scenario.
The constructed algorithms are afterwards is executed.

The property ensures we do not need to care about which implementation-aware algorithm we use since they all lead to the same simulation result.
However, the different algorithms naturally have different properties; the most interesting is probably the algorithm's execution time. 
We currently do not have the setup in place to explore this aspect.
Nevertheless, it could be interesting to explore this in the future.


\subsection{Design Space Exploration}
The Maude framework does also allow two different kinds of design space exploration using the integrated model checker.
This analysis enable a co-simulation practitioner to see how different model parameters changes the co-simulation result.
We describe the two independently even though they can be combined.
We start with describing how one can explore different instrumentations of a scenario.
Hereafter, we will describe how to explore different parameters for the simulation units.  

\subsubsection{Instrumentation of a Scenario}
As shown in \cite{Gomes2019,Oakes2021,hansen_verification_2021} contracts on the input ports can be used to achieve more accurate co-simulation results.
However, it is challenging for co-simulation practitioners to adapt their scenarios, especially because none of the current FMU-exporting tools provide this information on the SUs.

To ease the transition, we use Maude's model checker to explore the consequences of using different instrumentations of a scenario.

To explore different instrumentations of a scenario, we have defined the contract: \textit{noContract}.
The contract can be used if the user does not know the input contract.
We consider a scenario where the contract of at least one input is defined to be \textit{noContract} to be uninstrumented.

A scenario can only be simulated if it is instrumented.
A scenario is instrumented if no input in the scenario has a contract defined to be \textit{noContract}.


This allows Maude to play with different instrumentations of the scenario (different version of the function $\allreactivity$) such that for each input with the contract \textit{noContract} Maude tries to perform a simulation where the input is delayed and a simulation where the input is reactive.
The rewriting rules for this is given below:
\begin{alltt}
  \small
rl [instr-delayed]: 
findInstr(< SU1 : SU | inputs : (< I : Input | contract : noContract > IS) > C)
=>
findInstr(< SU1 : SU | inputs : (< I : Input | contract : delayed > IS) > C) .

rl [instr-reactive]: 
findInstr(< SU1 : SU | inputs : (< I : Input | contract : noContract > IS) > C)
=>
findInstr(< SU1 : SU | inputs : (< I : Input | contract : reactive > IS) > C) .

crl [remove-findInstr]: findInstr(CONF) => CONF 
  if allConstraintsDefined(CONF) .
\end{alltt}

The co-simulation practitioner can using the rewriting rule below experiment with different instrumentations of the scenario.

\begin{alltt}
  \small
  crl [findInstrumentation]: findContacts(INIT) => CONF
      if findInstr(INIT) => CONF
      \(\land\) empty == tarjan(CONF) *** No loops
      \(\land\) runAnyAlgorithm CONF => run: ORC on: FINAL with: SIMDATA
      \(\land\) shouldSatisfy(FINAL) .
  \end{alltt}

The rule generates all the different instrumentations of the scenario that satisfy certain properties.
It does so by first instrumenting the scenario to generate the instrumented scenario ``CONF''.
Hereafter it simulates the instrumented scenario ``CONF'' to check the resulting simulation.
The user can limit the explored instrumentations in desired ways in the above rule we for example specify that we are not interested instrumentations that generates algebraic loops.

Furthermore, the framework allows the co-simulation practitioner to specify concrete constraints on the simulation result result.
The constraints are above specified using the predicate \emph{shouldSatisfy}.

The different instrumentations are explored using Maude's search command.

\begin{alltt}
  \small
  search findContacts(Scenario) =>! C:Configuration .
\end{alltt}
  
This returns all the configurations/instrumentations of the scenarios that satisfy the desired properties defined in the above rule.

\subsubsection{Parameters of the Simulation Units}
An SU may have different parameters affecting its behavior.
This means that different parameters might lead to different simulation result.
Design space exploration is the process of exploring how the different valuations of the parameters affects the co-simulation result.
Design space exploration can therefore be used to find the optimal valuation of the parameters. 

The Maude framework allows design space exploration by letting the user specify the valuation a parameter as a set of possible valuations inside the \emph{choose}-operator as we shown below.
\begin{alltt}
  \small
< "tank" : SU | parameters : ("flow" |-> choose((< 1 >,< 2 >,< 30 >))) >
\end{alltt}

The \emph{choose}-operator non-deterministically picks a valuation in the set.
The framework uses the model checker to try out to run a simulation using all the different combinations of valuations for the desired properties.
This is done using the following rewriting rule.

\begin{alltt}
  \small
  crl [dse] : selectParams(UNITIALIZEDCONF) => CONF 
  if UNITIALIZEDCONF => CONF
  \(\land\) runAnyAlgorithm CONF => 
      run: < ALG : AlgData | Initialization : emptyList, 
      CosimStep : emptyList, Termination : emptyList > 
      on: FINALSTATE
      with: SIMULATIONDATA
  \(\land\) above10(FINALSTATE) .
\end{alltt}

The rule finds the valuations of the parameters that result in a simulation satisfying specific user-defined properties.
In the above rule, the valuation must ensure that the simulation satisfies the predicate \emph{above10}.

The parameters of the scenario can be found using a search command starting from a scenario with multiple valuations.
\begin{alltt}
  \small
  search selectParams(initConfDSE)  =>! C:Configuration .
\end{alltt}

\subsubsection{Combining Instrumentation and Parameter Search}
It is possible to combine the two searches to search both for the instrumentation and parameters of the scenario to obtain a given simulation.

This is simply done defining an uninstrumented scenario where some of the parameters can take different valuations.


\subsection{Limitation of the approach}
We have performed the analyses on different scenarios. 
The state-space explosion is, of course, also applying to our work. 
However, we have been able to find the correct instrumentation and the contracts on systems with \simon{Create big scenario and test} SUs.

Also, we do not currently support adaptive co-simulations \simon{Add reference to Emins work} where the instrumentation changes during the simulation. 
However, this is our impression that this feature trivial can be added.