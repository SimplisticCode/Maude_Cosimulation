
\section{Modeling Co-simulation Scenarios in Maude}
\label{sec:model}

This section describes how we model  individual SUs and their
composition in a co-simulation scenario in Maude.
%
Due to space limitations, we only provide fragments of our Maude
model. 
The entire model, including the synthesis and execution of co-simulation
algorithms (\Cref{sc:synthesize}) and the synthesis of
instrumentations and parameters (\Cref{sc:DSE})  is
available at \url{https://github.com/SimplisticCode/Co-simulation_WRLA} and consists of around 
1400 LOC. 

We formalize  co-simulation scenarios in an object-oriented style. The
 state is a term
 \texttt{\char123}$\mathit{SUs}\;\mathit{connections}\;\mathit{orchObjects}$\texttt{\char125}
 of sort \texttt{GlobalState}, 
 where $\mathit{SUs}$ is set of objects modeling simulation units,
$\mathit{connections}$ denote the port couplings, and
$\mathit{orchObjects}$ are two additional objects used
during synthesis and execution of co-simulation algorithms (see
\Cref{sc:synthesize}). 

% We illustrate our  framework using a system where a
% \emph{controller} 
% controls the water level of a \emph{water tank} with constant inflow
% of water,  by opening and closing a valve
% at the bottom of the tank. 
% The system can be modeled using  one SU for the tank and one for
% the controller, and has the architecture shown in
% \cref{fig:simpleexample}.

%\subsection{Simulation Units}
A simulation unit  is modeled as an object instance of the following
class: % a
                                % subsystem with parameters, ports, a
% state, and a time: 
\small
\begin{alltt}
class SU | time\,\,: Nat,                inputs\,\,: Configuration,
           outputs\,\,: Configuration,   canReject\,\,: Bool,
           fmistate\,\,: fmiState,       parameters\,\,: LocalState,
           localState\,\,: LocalState .
\end{alltt}
\normalsize

\noindent The attribute \texttt{time} denotes the time of the SU; 
\texttt{inputs} and \texttt{outputs} denote the objects modeling the
SU's input and output 
ports; \texttt{canReject}  is \texttt{true} if  the SU
implements error estimation (i.e., is an element of the set
$\mayReject$);  
\texttt{fmistate} denotes the \emph{simulation mode}
(see~\cite{FMI2014}) of the SU; \texttt{localState} denotes the SU's internal
state; and \texttt{parameters} denotes the values of the SU's
parameters. 

Input and output ports are modeled as  instances of the following
classes:

\small
\begin{alltt}
class Port | value\,:\,\,FMIValue,\,time\,:\,\,Nat,\,status\,:\,\,PortStatus,\,type\,:\,\,FMIType\,. 
class Input | contract : Contract .
class Output | dependsOn : OidSet .     
subclasses Input Output < Port .
\end{alltt}
\normalsize

\noindent % The attributes
\texttt{value} and \texttt{time} denote, respectively, the
value of the port and the time of its last set/get operation; 
\texttt{status} is \texttt{true}  if the port was
updated  at the current time;   \texttt{contract} denotes the
input port's instrumentation (\texttt{delayed} or
\texttt{reactive}); and 
\texttt{dependsOn} denotes the set of inputs that feed
through to the output port. 

\begin{example}
  We illustrate our  framework using a system where a
\emph{controller} 
controls the water level of a \emph{water tank} with constant inflow
of water,  by opening and closing a valve
at the bottom of the tank. 
The system can be modeled using  one SU for the tank and one for
the controller, and has the architecture shown in
\cref{fig:simpleexample}.
%
The water tank  (in its initial state)  is modeled as an object
  
\scriptsize
\begin{alltt}
< "tank" : SU | parameters : ("flow" |-> <\,5\,>),  localState : ("waterlevel" |-> <\,0\,>),
                inputs : (< "valveState" : Input |\,value\,\,:\,\,<\,0\,>, time\,\,:\,\,0,\,contract\,\,:\,\,delayed >),
                outputs : (< "waterlevel" : Output | value : <\,0\,>, time : 0,
                                                     status : Undef, dependsOn : empty >)
                time : 0,  canReject : false >
\end{alltt}
\normalsize

\noindent The \texttt{tank} has one delayed input port and one output
port, and the local state indicates that the tank is empty.  
The parameter \texttt{flow} defines the amount of water that flows
into the tank per time unit.
\end{example}

To formalize the behaviors of an SU we formalize the operations
\texttt{set}, \texttt{get}, and \texttt{step} in
\cref{def:fmu}. 
For example, the $\fget{}$ operation that updates the \texttt{time}
and \texttt{status}  of a set
of output ports is formalized as follows:\footnote{We do not show
  variable declarations, but follow the convention that variables are
  written with capital letters.} 

\small
\begin{alltt}
  op getAction : Object OidSet -> Object .
  eq getAction(< SU1 : SU | >, empty) = < SU1 : SU | > .
  eq getAction(< SU1 : SU | time : T, 
                            outputs : (< O : Output | > OS) >, (O , P)) = 
     getAction(< SU1 : SU | outputs : 
                     (< O : Output | time : T, status : Def > OS) >, P) .
\end{alltt}
\normalsize

% The operator iterates over a defined set of output ports.
% The \textit{time} and \textit{state} of the output port is update to
% the time $T$ and the input is marked as defined \textit{Def}. 

%The precondition ($\fpreget{}$) is omitted from the operation
%\emph{getAction} since it is included in the functions calling it. 
%The functions $\fget{}$ and $\fset{}$ are similar for all SUs
%described in the framework. 

The concrete behavior of an SU is given by defining its $\fdoStep{}$
function:

\begin{example}
The following definition of the \texttt{step} function in our running
example says that  the water level of the tank changes according to
step duration \texttt{STEP}, the parameter \texttt{flow}, and the
state (\texttt{value}) of the input \texttt{valve}:

\scriptsize
\begin{alltt}
eq step(< "tank" : SU | time : T, parameters : ("flow" |-> <\,FLOW\,>), 
                        inputs : < "valve" : Input | value : <\,STATE\,> >, 
                        outputs : < "waterlevel" : Output | time : T >,
                        localState : ("waterlevel" |-> <\,LEVEL\,>) >,
        STEP) = 
  if STATE == 1 then    \emph{--- valve is open}
    < "tank" : SU | time : (T\,+\,STEP), localState : ("waterlevel" |-> <\,0\,>),
          outputs : < "waterlevel" : Output | value\,\,:\,\,<\,0\,>, time\,\,:\,\,(T\,+\,STEP), status\,\,:\,\,Undef > >
  else                  \emph{--- valve is closed}
    < "tank" : SU | time : (T\,+\,STEP),  localState : ("waterlevel" |-> <\,LEVEL\,+\,(STEP\,*\,FLOW)\,>), 
                    outputs : < "waterlevel" : Output | value : < LEVEL + (STEP * FLOW) >, 
                                                        time : (T + STEP), status : Undef > > 
  fi .
\end{alltt}
\normalsize
\end{example}

%A scenario is a collection of SUs composed using a set of couplings/connections.
A connection/coupling connecting the output port $o$ of SU $\mathit{su}_1$ to
the input port $i$ of SU $\mathit{su}_2$ is represented by the term
$\mathit{su}_1$ \texttt{!} $o$ \texttt{==>}
$\mathit{su}_2$ \texttt{!} $i$.   %  of a subsort
% \texttt{Connection} of sort \texttt{Configuration}.


% ports connects an output with an input:
% \begin{alltt}
%   \small
% op _==>_ : EPortId EPortId -> Connection [ctor] .
% subsort Connection < Configuration .
% \end{alltt}

We define scenarios by defining constants \texttt{simulationUnits}
and \texttt{externalConnection} that denote, resp.,  the
simulation unit objects and  their connections. 

\begin{example}\label{ex:simulationunits}
The SUs and the couplings in our  example are defined as
follows:

\scriptsize
\begin{alltt}
eq simulationUnits = 
   < "tank" : SU | parameters : ("flow" |-> <\,100\,>),  localState : ("waterlevel" |-> <\,0\,>),
                   time : 0,  fmistate : Instantiated, canReject : false, 
                   inputs : (< "valveState" : Input | value : <\,0\,>, type : integer, time : 0,
                                                      contract : delayed, status : Undef >), 
                   outputs : (< "waterlevel" : Output | value : <\,0\,>, type\,:\,integer, time\,:\,0,
                                                        status : Undef, dependsOn : empty >)\,>
   < "ctrl" : SU | parameters : (("high" |-> <\,5\,>) , ("low" |-> <\,0\,>)), canReject : false, 
                   localState : ("valve" |-> <\,false\,>), fmistate : Instantiated, time : 0, 
                   inputs : (< "waterlevel" : Input | value : <\,0\,>, type : integer, time : 0,
                                                      contract : reactive, status : Undef >), 
                   outputs : (< "valveState" : Output | value : <\,0\,>, type\,:\,integer, time\,:\,0,
                                                        status : Undef,\,dependsOn : empty\,>)\,>\,.

eq externalConnection = ("tank" ! "waterlevel" ==> "ctrl" ! "waterlevel") 
                        ("ctrl" ! "valveState" ==> "tank" ! "valveState") .
\end{alltt}
\normalsize
\end{example}

%A scenario is described as a multiset of SUs and connections linking the ports.
%A simulation is instantiated as a \textit{GlobalState} using the equation:
The constant \texttt{setup} defines the initial state, and adds
appropriate initialized orchestration objects to the scenario:

\small
\begin{alltt}
op setup : -> GlobalState .
ceq setup = \char123INIT\char125
  if SCENARIO := externalConnection simulationUnits
  /\char92 validScenario(SCENARIO)
  /\char92 LOOPS := tarjan(SCENARIO)
  /\char92 NeSUIDs := getSUIDsOfScenario(SCENARIO)
  /\char92 INIT := calculateSNSet(SCENARIO OData(1,LOOPS, NeSUIDs)) .
\end{alltt}
\normalsize

\noindent The function \texttt{validScenario} checks whether all inputs are coupled and that no input has two sources.   
The function \texttt{tarjan} returns (a possibly empty) set  of algebraic loops in the scenario by searching for non-trivial strongly connected components in
the graph constructed using the rules  in \cite{Gomes2019c}.   
The function \texttt{getSUIDsOfScenario} returns  the set of all SU
identifiers. Finally, \texttt{calculateSNSet} checks if step negotiation 
should be applied in the simulation of the scenario, and generates a
global initial state with orchestration objects that store information
about the discovered algebraic loops and whether step negotiation is
needed.