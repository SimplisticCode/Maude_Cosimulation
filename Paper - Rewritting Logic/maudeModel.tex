\section{Formal model of co-simulation in Maude}
This section describes how co-simulation is modelled in Maude.
We start with describing the SUs before addressing their composition in a co-simulation.


\subsubsection{Modeling SU behaviour}
The behaviour of an SU is modelled by defining a custom version of the $\fdoStep{C}$ for each of the SUs. 
The $\fdoStep{C}$ can be made both non-deterministic using rewriting rules and deterministic using equations.


We place the delayed/reactive information on the Input ports in Maude. This is denoted as the \textit{Contract} of the input.
Likewise, the feedthrough information is a property on the output ports.  

The SUs are of the type 
\begin{lstlisting}
class SU |
  path : String, 
  time : Nat, 
  inputs : Configuration, 
  outputs : Configuration,
  canReject : Bool, 
  fmistate : fmiState,
  parameters : LocalState,
  localState : LocalState .
\end{lstlisting}




\subsection{Orchestrator}
The orchestrator running the simulation stores some information to exchange values between the SUs, decide how far the simulation units should step.
This data is referred to as simulation data.

\subsection{The co-simulation algorithm}
The co-simulation algorithm is divided into three distinct components: 
\begin{compactitem}
  \item The initialization procedure
  \item The cosim-step procedure 
  \item The termination
\end{compactitem}
All the three components are of the type ActionList describing the actions to be performed in the co-simulation.
