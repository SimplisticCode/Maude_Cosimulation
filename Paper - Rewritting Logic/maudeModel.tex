\section{Formal model of co-simulation in Maude}
This section describes how co-simulation is modelled in Maude.
We start with describing the SUs before addressing their composition in a co-simulation.

\begin{definition}[Simulation Unit]\label{def:fmu}
  An SU with identifier $c$ is represented by the tuple
  $$\tuple{\stateset{c}, \inputs{c}, \outputs{c}, \parameters{c}, \fset{c}, \fget{c}, \fdoStep{c}},$$
  where:
  \begin{compactitem}
    \item $\stateset{c}$ represents the state space.
    \item $\inputs{c}$ and $\outputs{c}$ the set of input and output variables, respectively. 
    \item $\parameters{c}$ is the set of parameters.
    \item $\fset{c} : \stateset{c} \times \inputs{c} \times \valuesExchanged \to \stateset{c}$ and $\fget{c}: \stateset{c} \times \outputs{c} \to \valuesExchanged$ are functions to set the inputs and get the outputs, respectively. The set of values exchanged between the input/output variables is abstracted as $\valuesExchanged$, a set of type: $\tuple{t, \values}$, where $\values$ denotes the value obtained at a given output port and $t: \timebase$ denotes the timestamp of $c$ when the value was obtained.
    \item $\fdoStep{c}: \stateset{c} \times \stepbase \to \stateset{c} \times \stepbase $ is a function that instructs the SU to compute its state after a given time duration. If an SU is in state $\stateafter{c}{t}$ at time $t$, $(\stateafter{c}{t+h}, h) = \fdoStep{c}(\stateafter{c}{t}, H)$ approximates the state $\stateafter{c}{t+h}$ of the corresponding model at time $t+h$, where $h \leq H$. 
  \end{compactitem}
\end{definition}

\Cref{def:fmu} is inspired by \cite{Broman2013,Gomes2019c,thrane2021} and represents a symbolic version of an SU. 

The state of SU $A$ at time $t$ is denoted $\stateafter{A}{t}$.
We assume the last value set on an input/output port can be inspected, for example, the value of input $\inputvar{x}$ could be $\inputvar{x} = \tuple{t, v_x}$, where $t$ is the timestamp when the value $v_x$ set on $\inputvar{x}$ was obtained.
The function $\fdoStep{c}$ returns a step size because some SUs implement error estimation and may conclude that taking a step size of $H$ will result in an intolerable error meaning the SU takes a smaller step than planned.

\subsubsection{Modeling SU behaviour}
The behaviour of an SU is modelled by defining a custom version of the $\fdoStep{C}$ for each of the SUs. 
The $\fdoStep{C}$ can be made both non-deterministic using rewriting rules and deterministic using equations.



\begin{definition}[Scenario]\label{def:cosim_scenario}
  A scenario is a structure $\tuple{\fmus, \coupling, \mayReject, \allfeedthroughs, \allreactivity, \alldelayed}$ where each identifier $c \in \fmus$ is associated with an SU, as defined in \cref{def:fmu}, and $\coupling(u)=y$ means that the output $y$ is connected to input $u$.
  Let $\allinputs = \bigcup_{c \in \fmus} \inputs{c}$ and $\alloutputs = \bigcup_{c \in \fmus} \outputs{c}$, then $\coupling : \allinputs \to \alloutputs$. 
  $\mayReject \subseteq \fmus$ denotes the SUs that implement error estimation. 
  The set of reactive components,
  $\allreactivity = \bigcup_{c \in \fmus} \reactivity{c}$, where $\reactivity{c}(\inputvar{c}) = \true$ means the function $\fdoStep{c}$ assumes that the input $\inputvar{c}$ comes from an SU that has advanced forward relative to SU $c$.  
The set of delayed components,
  $\alldelayed = \bigcup_{c \in \fmus} \neg \reactivity{c}$, where $\reactivity{c}(\inputvar{c}) = \false$ means the function $\fdoStep{c}$ assumes that the input $\inputvar{c}$ comes from an SU that is at the same time as SU $c$. 
 Finally, the set of feed-through components, $\allfeedthroughs = \bigcup_{c \in \fmus} \feedthrough{c}$, where the input $\inputvar{c} \in \inputs{c}$ feeds through to output $\outputvar{c} \in \outputs{c}$, that is, $(\inputvar{c},\outputvar{c}) \in \feedthrough{c}$, when there exists $v_1, v_2 \in \valuesExchanged$ and $\state{c} \in \stateset{c}$, such that
  $\fget{c} (\fset{c}(\state{c}, \inputvar{c}, v_1), \outputvar{c}) \neq \fget{c} (\fset{c}(\state{c}, \inputvar{c}, v_2), \outputvar{c}).$
\end{definition}  

We have decided to define inputs in Maude as something having a contract:

The outputs contain the feedthrough information:

The SUs are of the type 
\begin{lstlisting}
class SU |
  path : String, 
  time : Nat, 
  inputs : Configuration, 
  outputs : Configuration,
  canReject : Bool, 
  fmistate : fmiState,
  parameters : LocalState,
  localState : LocalState .
\end{lstlisting}



\subsection{Orchestrator}
The orchestrator running the simulation needs to store some information to exchange values between the SUs, decide how fair the simulation units should step.
This data is referred to as simulation data.

\subsection{The co-simulation algorithm}
The co-simulation algorithm is divided into three distinct components: 
\begin{compactitem}
  \item The initialization procedure
  \item The cosim-step procedure 
  \item The termination
\end{compactitem}
All the three components are of the type ActionList describing the actions to be performed in the co-simulation.