\section{Model checking Analysis of co-simulation}
The non-deterministic rewrites rule in Maude allows exploration multiple instrumentations, 


\begin{itemize}
    \item We have small scenarios at the moment.
    \item Test this on a bigger example - synthesize one with 6,8,10 SUs.
\end{itemize}


\subsection{Orchestration Algorithms}
All constructed orchestration algorithms of a instrumented scenario produces a deterministic co-simulation result.




\subsection{Design Space Exploration}
Design Space Exploration is the 

\subsubsection{Instrumentation of Scenario}
The beginning adaption of contracts for input ports to achieve more accurate co-simulation results. 
A prominent problem in this domain is that   

Find instrumentations leading to a desired co-simulation result - simple example. 

Furthermore, it is possible to restrict the set of explored instrumentations to a set satisfying specific properties such as no algebraic loops.
This restriction is desireable in many applications where the orchestration engine does support for these and also for simulations of real-time systems.

\subsubsection{Parameters of the Simulation Units}
A simulation unit has 



We can potentially use SMT-solving later to help us with the design space exploration.


We have performed the different analyses on different scenarios. 
The state space explosion is of course also applying to our work, however we have been able to find the correct instrumentation and the contracts on systems with XX SUs.

Our test examples include the industrial case study original from Boeing presented in (Add reference to Gomes). 