\section{Synthesizing and Executing Co-simulation Algorithms}\label{sc:synthesize}
This section describes how co-simulation algorithms for a given scenario can be synthesized and then executed in in Maude.

\subsection{The Orchestration of a Co-simulation}
The orchestration is responsible for orchestrating the behavior of the scenario. 
%is responsible for the simulation of a scenario.
The simulation requires a co-simulation algorithm to be synthesized and executed.
The orchestration of a scenario requires that we keep track of the co-simulation algorithm and additional data about the current execution of the algorithm.

The orchestration is modeled by instances of the following classes \texttt{SimData} and \texttt{AlgoData}:

\small
\begin{alltt}
class SimData | SNSet : OidSet,           defaultStepSize : NzNat,
          actualStepSize : NzNat,         unsolvedSCC : AlgebraicLoopSet, 
          solvedSCC : AlgebraicLoopSet,   guessOn : PortSet,
          values : PortValueMap,          simulationTime : Nat,
          suids : NeOidSet .
\end{alltt}
\normalsize

The class \texttt{SimData} contains data needed for the orchestration during the simulation.
The attribute \texttt{SNSet} denotes the set $\mayReject$ (see \cref{def:cosim_scenario}); \texttt{defaultStepSize} is the default step duration of the simulation; while the attribute \texttt{actualStepSize} is the negotiated step duration.
The attributes \texttt{actualStepSize} and \texttt{defaultStepSize} are equal if $\mayReject = \emptyset$.
The attributes \texttt{unsolvedSCC} and \texttt{solvedSCC} respectively denote the solved and unsolved algebraic loops.
The attribute \texttt{guessOn} is the set of ports, which is used to solve algebraic loops using the reduction technique described in \cite{thrane2021}; \texttt{values} is a map linking an input port with a value. 
The orchestration uses \texttt{values} to track which values it has obtained but not set on an input port. 
The attribute \texttt{simulationTime} describes the current time of the simulation, and \texttt{suids} denotes the identifiers of the SUs.

\small
\begin{alltt}
class AlgoData | CosimStep : ActionList,   Initialization : ActionList, 
                 Termination : ActionList, endTime : NzNat .
\end{alltt}
\normalsize

The class \texttt{AlgoData} describes the co-simulation algorithm, which is divided into three parts.
The attributes \texttt{Initialization} and \texttt{Termination} denote the initialization procedure and termination procedure, respectively.
The attribute \texttt{CosimStep} denotes the co-simulation step procedure that the orchestration applies until it reaches the end time of the simulation, which is denoted by the attribute \texttt{endTime}.
All elements of the algorithm are of the sort \texttt{ActionList}, which is a list of SU operations:

\small
\begin{alltt}
ops Set Get Step Save : -> ActionType [ctor] . 
op portEvent:_SU:_PId:_ : ActionType SUID OidSet -> Action [ctor] .
subsort Action < ActionList .
op emptyList : -> ActionList [ctor] .
op _;_ : ActionList ActionList -> ActionList [ctor assoc id: emptyList] .
\end{alltt}
\normalsize

We have omitted the actions to handle complex scenarios.

% The orchestrator in Maude has the following two objectives.
% \begin{compactenum}
%   \item Synthesize a co-simulation algorithm
%   \item Execute the co-simulation algorithm.
% \end{compactenum}

\subsection{Synthesis of Co-simulation Algorithms}
The framework synthesizes co-simulation algorithms for a scenario $\scenario$ by (1) performing and recording all possible SU actions and (2) when searching for consistent reachable final states.
Any sequence of SU actions leading to such a state gives us a co-simulation algorithm.

A number of rewrite rules model the different SU actions.
For example, the following rewrite rule describes a $\fget{}$ operation:
\small
\begin{alltt}
crl [get-syn] :
    < SU1 : SU | fmistate : Simulation, inputs : IS, 
                  outputs : (< O : Output | time : T, status : Undef, 
                                            value : V, dependsOn : FT > OS) > 
    (SU1 ! O ==> SU2 ! I)
    < OCH : SimData | values : PV > 
    < ALG : AlgoData | CoSimStep : ALGO >
    => 
    getAction(< SU1 : SU | >, SU1 ! O)
    (SU1 ! O ==> SU2 ! I)
    < OCH : SimData | values : insert((SU2 ! I), < T ; V >, PV) >  
    < ALG : AlgoData | CoSimStep : (ALGO ; EVENT) >
  if feedthroughSatisfied(FT, IS, T) 
    /\char92 EVENT := portEvent: Get SU: SU1 PId: O . 
\end{alltt}
\normalsize

A value \texttt{V} can be obtained from the output \texttt{O} of an SU \texttt{SU1} if the current state satisfy all the feed-through constraints \texttt{FT} of the output \texttt{O}, which is checked by the function \texttt{feedthroughSatisfied}.
The rule updates the output \texttt{O} using the operation \texttt{getAction} and inserts the output's value and time \texttt{< T ; V >} in the map \texttt{values}.
Finally, it adds the performed action \texttt{portEvent: Get SU: SU1 PId: O} to its record of performed actions \texttt{CoSimStep}.

All such ``synthesis'' rules in our model follow the same pattern.
They rewrite the scenario while remembering how they did it.

We can synthesize a co-simulation algorithm by starting from a consistent initial state and explore how a consistent final state can be established.
An algorithm for a given scenario is, therefore, synthesized using the following rewrite rule:

\small
\begin{alltt}
crl [getAlgortihm]: \{INIT\} => \{getOrchestrator(FINALSTATE)\}
if isInitialState(INIT)
  /\char92 LOOPS := tarjan(INIT)
  /\char92 SUIDsNE := getSUIDsOfScenario(INIT)
  /\char92 SIMDATA := initialOrchestrationData(1,LOOPS,SUIDsNE)
  /\char92 CONF := calculateSNSet(INIT) SIMDATA initialAlgorithmData(1)
  /\char92 \{CONF\} => \{FINALSTATE\} 
  /\char92 allSUsinUnloaded(SUIDsNE, FINALSTATE) .
\end{alltt}
\normalsize

The rule \texttt{getAlgorithm} checks if the scenario \texttt{INIT} is in a suitable initial state using the predicate \texttt{isInitialState}.
Then we construct an initial simulation configuration \texttt{CONF} as in \cref{sec:model}.
It rewrites the state \texttt{\{CONF\}} to \texttt{\{FINALSTATE\}} using the nested rewriting rule 
The key condition that does all the work is the rewrite condition \texttt{\{CONF\} => \{FINALSTATE\}}, which searches all states reachable from \texttt{CONF} until it finds one such state \texttt{FINALSTATE} that satisfies the property \texttt{allSUsinUnloaded}, which ensures that all SUs have been properly simulated and unloaded.
The function \texttt{getOrchestrator} extracts the synthesized co-simulation algorithm from this final state.

\begin{example}
  Extracting an algorithm for the scenario described by \cref{ex:simulationunits} yields six different algorithms and is performed using the command:

  \small
  \begin{alltt}
    search scenario => FINALSTATE::GlobalState .
  \end{alltt}
  \normalsize

%   \small
% \begin{alltt}
% Solution 1
% {< "Algorithm" : AlgorithmData | 
% CosimStep :
%   ((stepEvent: "tank" StepSize: 1);
%   (portEvent: Get SU: "tank" PId: "waterlevel");
%   (portEvent: Set SU: "ctrl" PId: "waterlevel");
%   (stepEvent: "ctrl" StepSize: 1);
%   (portEvent: Get SU: "ctrl" PId: "valveState");
%   portEvent: Set SU: "tank" PId: "valveState"),
% Initialization :
%   ((fmiEvent: CreateExp SUs:("ctrl", "tank"));(
%   fmiEvent: EnterInitialization SUs:("ctrl", "tank"));
%     (portEvent: Get SU: "ctrl" PId: "valveState");
%     (portEvent: Get SU: "tank" PId: "waterlevel");
%     (portEvent: Set SU: "ctrl" PId: "waterlevel");
%     (portEvent: Set SU: "tank" PId: "valveState");
%   fmiEvent: ExitInitialization SUs:("ctrl", "tank")),
% Termination :
%   ((fmiEvent: Terminate SUs:("ctrl", "tank"));
%   (fmiEvent: FreeInstance SUs:("ctrl", "tank"));
%   fmiEvent: Unload SUs:("ctrl", "tank")),
% endTime : 1 >}

% Solution 2
% {< "Algorithm" : AlgorithmData | 
% CosimStep :
%   ((stepEvent: "tank" StepSize: 1);
%   (portEvent: Get SU: "tank" PId: "waterlevel");
%   (portEvent: Set SU: "ctrl" PId: "waterlevel");
%   (stepEvent: "ctrl" StepSize: 1);
%   (portEvent: Get SU: "ctrl" PId: "valveState");
%   portEvent: Set SU: "tank" PId: "valveState"),
% Initialization :
%   ((fmiEvent: CreateExp SUs:("ctrl", "tank"));(
%   fmiEvent: EnterInitialization SUs:("ctrl", "tank"));
%     (portEvent: Get SU: "ctrl" PId: "valveState");
%     (portEvent: Get SU: "tank" PId: "waterlevel");
%     (portEvent: Set SU: "tank" PId: "valveState");
%     (portEvent: Set SU: "ctrl" PId: "waterlevel");
%   fmiEvent: ExitInitialization SUs:("ctrl", "tank")),
% Termination :
%   ((fmiEvent: Terminate SUs:("ctrl", "tank"));
%   (fmiEvent: FreeInstance SUs:("ctrl", "tank"));
%   fmiEvent: Unload SUs:("ctrl", "tank")),
% endTime : 1 >}


% Solution 3
% {< "Algorithm" : AlgorithmData | 
% CosimStep :
%   ((stepEvent: "tank" StepSize: 1);
%   (portEvent: Get SU: "tank" PId: "waterlevel");
%   (portEvent: Set SU: "ctrl" PId: "waterlevel");
%   (stepEvent: "ctrl" StepSize: 1);
%   (portEvent: Get SU: "ctrl" PId: "valveState");
%   portEvent: Set SU: "tank" PId: "valveState"),
% Initialization :
%   ((fmiEvent: CreateExp SUs:("ctrl", "tank"));(
%   fmiEvent: EnterInitialization SUs:("ctrl", "tank"));
%     (portEvent: Get SU: "ctrl" PId: "valveState");
%     (portEvent: Set SU: "tank" PId: "valveState");
%     (portEvent: Get SU: "tank" PId: "waterlevel");
%     (portEvent: Set SU: "ctrl" PId: "waterlevel");
%   fmiEvent: ExitInitialization SUs:("ctrl", "tank")),
% Termination :
%   ((fmiEvent: Terminate SUs:("ctrl", "tank"));
%   (fmiEvent: FreeInstance SUs:("ctrl", "tank"));
%   fmiEvent: Unload SUs:("ctrl", "tank")),
% endTime : 1 >}

% Solution 4
% {< "Algorithm" : AlgorithmData | 
% CosimStep :
%   ((stepEvent: "tank" StepSize: 1);
%   (portEvent: Get SU: "tank" PId: "waterlevel");
%   (portEvent: Set SU: "ctrl" PId: "waterlevel");
%   (stepEvent: "ctrl" StepSize: 1);
%   (portEvent: Get SU: "ctrl" PId: "valveState");
%   portEvent: Set SU: "tank" PId: "valveState"),
% Initialization :
%   ((fmiEvent: CreateExp SUs:("ctrl", "tank"));(
%   fmiEvent: EnterInitialization SUs:("ctrl", "tank"));
%     (portEvent: Get SU: "tank" PId: "waterlevel");
%     (portEvent: Get SU: "ctrl" PId: "valveState");
%     (portEvent: Set SU: "ctrl" PId: "waterlevel");
%     (portEvent: Set SU: "tank" PId: "valveState")
%   fmiEvent: ExitInitialization SUs:("ctrl", "tank")),
% Termination :
%   ((fmiEvent: Terminate SUs:("ctrl", "tank"));
%   (fmiEvent: FreeInstance SUs:("ctrl", "tank"));
%   fmiEvent: Unload SUs:("ctrl", "tank")),
% endTime : 1 >}

% Solution 5
% {< "Algorithm" : AlgorithmData | 
% CosimStep :
%   ((stepEvent: "tank" StepSize: 1);
%   (portEvent: Get SU: "tank" PId: "waterlevel");
%   (portEvent: Set SU: "ctrl" PId: "waterlevel");
%   (stepEvent: "ctrl" StepSize: 1);
%   (portEvent: Get SU: "ctrl" PId: "valveState");
%   portEvent: Set SU: "tank" PId: "valveState"),
% Initialization :
%   ((fmiEvent: CreateExp SUs:("ctrl", "tank"));(
%   fmiEvent: EnterInitialization SUs:("ctrl", "tank"));
%     (portEvent: Get SU: "tank" PId: "waterlevel");
%     (portEvent: Get SU: "ctrl" PId: "valveState");
%     (portEvent: Set SU: "tank" PId: "valveState");
%     (portEvent: Set SU: "ctrl" PId: "waterlevel");
%   fmiEvent: ExitInitialization SUs:("ctrl", "tank")),
% Termination :
%   ((fmiEvent: Terminate SUs:("ctrl", "tank"));
%   (fmiEvent: FreeInstance SUs:("ctrl", "tank"));
%   fmiEvent: Unload SUs:("ctrl", "tank")),
% endTime : 1 >}

% Solution 6
% {< "Algorithm" : AlgorithmData | 
% CosimStep :
%   ((stepEvent: "tank" StepSize: 1);
%   (portEvent: Get SU: "tank" PId: "waterlevel");
%   (portEvent: Set SU: "ctrl" PId: "waterlevel");
%   (stepEvent: "ctrl" StepSize: 1);
%   (portEvent: Get SU: "ctrl" PId: "valveState");
%   portEvent: Set SU: "tank" PId: "valveState"),
% Initialization :
%   ((fmiEvent: CreateExp SUs:("ctrl", "tank"));(
%   fmiEvent: EnterInitialization SUs:("ctrl", "tank"));
%   (portEvent: Get SU: "tank" PId: "waterlevel");
%   (portEvent: Set SU: "ctrl" PId: "waterlevel");
%   (portEvent: Get SU: "ctrl" PId: "valveState");
%   (portEvent: Set SU: "tank" PId: "valveState");
%   fmiEvent: ExitInitialization SUs:("ctrl", "tank")),
% Termination :
%   ((fmiEvent: Terminate SUs:("ctrl", "tank"));
%   (fmiEvent: FreeInstance SUs:("ctrl", "tank"));
%   fmiEvent: Unload SUs:("ctrl", "tank")),
% endTime : 1 >}
% \end{alltt}
% \normalsize
%We can see that all the synthesized algorithms have the same co-simulation step that is similar to \cref{alg:algorithm_3}.
\end{example}

Our framework also supports synthesizing co-simulation algorithms for complex co-simulation scenarios see the online model for details.

Many SU actions can happen at the same time, which means that multiple correct algorithms often can be synthesized for the same scenario.

For example, there exists six different algorithms to simulate the water tank example.

\subsection{Executing Co-Simulation Algorithms}
This section describes how we can execute the generated co-simulation algorithms.

The state of such an execution is a term: \texttt{run:} \emph{algorithm} \texttt{on:} \emph{scenario} \texttt{with:} \emph{simulationData} with the constructor:

\small
\begin{alltt}
op run:_on:_with:_ : Object Configuration Object -> SimState [ctor]. 
\end{alltt}
\normalsize

A co-simulation algorithm is executed by sequentially performing its actions, by starting with performing all actions in the \texttt{Initialization}, then performing all actions of the \texttt{CosimStep} before executing all the actions of the \texttt{Termination}. 

For example, the following rules shows the execution of the first action \texttt{(action: Get SU: SU1 PId: O)} in the list \texttt{CosimStep}:

\scriptsize
\begin{alltt}
crl [get-exec] : 
    run: < ALG : AlgoData | CosimStep : \emph{(action: Get SU: SU1 PId: O)} ; ALGO >
    on: CONF
      < SU1 : SU | inputs : IS,
                   outputs : (< O : Output | time : T, value : V, dependsOn : FT > OS) > 
      ( SU1 ! O ==> SU2 ! INPUT)
    with: < OCH : SimData | values : PV >
    => 
    run: < ALG : AlgoData | CosimStep : ALGO >
    on: CONF getAction(< SU1 : SU | >, SU1 ! O) ( SU1 ! O ==> SU2 ! INPUT)
    with: < OCH : SimData | values : insert((SU2 ! INPUT), < T ; V >, PV) > 
 if feedthroughSatisfied(FT, IS, T) .
\end{alltt}
\normalsize
%The rule performs the first action \texttt{(action: Get SU: SU1 PId: O)} .
%The action state that we should get the value of output \texttt{O} of SU \texttt{SU1}.
%The rule ensures that all feed-through requirements are satisfied.
%The result of the rule is that the value \texttt{V} is saved in the map \texttt{values}, the output is updated, and that the performed action is removed from the list of actions defined by the \texttt{CoSimStep}.

%The simulation finishes when all the actions have been exhausted, and no more rewritings can occur.

We can combine algorithm synthesis and execution into the following rewrite rule, so that rewriting the term \texttt{runAnyAlgorithm} \emph{scenario} synthesizes \emph{and} executes a co-simulation algorithm the for co-simulation scenario \emph{scenario}:

\small
\begin{alltt}
crl [runAlg] : runAnyAlgorithm INIT => run: ORC on: INIT with: SIMDATA
  if LOOPS := tarjan(INIT)
  /\char92 SUIDsNE := getSUIDsOfScenario(INIT)
  /\char92 SIMDATA := initialOrchestrationData(1,LOOPS,SUIDsNE)
  /\char92 ALGO := initialAlgorithmData(1)
  /\char92 CONF := calculateSNSet(INIT ALGO) SIMDATA 
  /\char92 \{CONF\} => \{ FINALSTATE \} 
  /\char92 ORC := getOrchestrator(FINALSTATE)
  /\char92 allSUsinUnloaded(SUIDsNE, FINALSTATE) .
\end{alltt}
\normalsize

This rule is similar to the rule \texttt{getAlgortihm}, and also extracts the resulting algorithm \texttt{ORC} and simulation data \texttt{SIMDATA}.

\begin{example}\label{ex:algoexecution}
The water tank scenario described in \cref{ex:simulationunits} (\texttt{waterTankScenario} below) can be simulated by rewriting the term:
\small
\begin{alltt}
  frew (runAnyAlgorithm waterTankScenario).
\end{alltt}
\normalsize
The command returns the final simulation state:
\simon{Make inline using \,}
\scriptsize
\begin{alltt}
run: < "Algorithm" : AlgorithmData | CosimStep : \emph{emptyList}, Initialization : emptyList,
                                     Termination : emptyList, endTime : 1 > 
on: ("tank" ! "waterlevel" ==> "ctrl" ! "waterlevel") 
    ("ctrl" ! "valveState" ==> "tank" ! "valveState")
    < "ctrl" : SU | canReject : false, fmistate : Unloaded, time : 1,
                    inputs : < "waterlevel" : Input | contract : reactive, status : Def, 
                                                      time : 1, type : integer, 
                                                      value : < 100 > >,
                    outputs : < "valveState" : Output | dependsOn : empty, status : Def, 
                                                        time : 1, type : integer, 
                                                        value : < 1 > >,
                    localState : "valve" |-> < true >,    
                    parameters :"high" |-> < 5 >, "low" |-> < 0 >  > 
    < "tank" : SU | canReject : false, fmistate : Unloaded, time : 1,
                    inputs : < "valveState" : Input | contract : delayed, status : Def,
                                                      time : 1, type : integer, value : < 1 > >,
                    outputs : < "waterlevel" : Output | dependsOn : empty, status : Def,
                                                        time : 1, type : integer, 
                                                        value : < 100 > >,
                    localState : \emph{"waterlevel" |-> < 100 >},
                    parameters : "flow" |-> < 100 > > 
with: < "Orchestrator" : SimulationData | SNSet : empty,       actualStepSize : 1, 
                                          defaultStepSize : 1, guessOn : empty,
                                          simulationTime : 1,  solvedSCC : empty, 
                                          unsolvedSCC : empty, values : empty,
                                          suids :("ctrl", "tank") >
\end{alltt}
\normalsize

\end{example}


\subsection{Checking Confluence of Synthesized Co-simulation Algorithms}
All valid co-simulation algorithms for the same scenario should give the same simulation result.
We can use Maude to check whether all generated co-simulation algorithms for a given scenario \emph{scenario} result in the same final state using the following search command:

\small
\begin{alltt}
  search (runAnyAlgorithm \emph{scenario})  =>! S:SimState . 
  \end{alltt}
\normalsize

This search command synthesizes and then executes all co-simulation algorithms for the scenario \emph{scenario}. 
%Afterwards, it rewrites the scenario using the algorithm until it reaches a final state \texttt{S} that cannot be rewritten any further.
For the water tank scenario in \cref{ex:simulationunits}, the search produces an single result, which means that all synthesized algorithms give the same result.

%Confluence of the synthesized co-simulation algorithms is essential; it ensures that all correct algorithms lead to the same simulation result.

Even though different algorithms give the same final result, they can still have different properties, such as memory overhead and execution time. 
Our framework currently does not allow us to explore this aspect, since this relies on concrete SUs of the system.
Nevertheless, we believe that this aspect could be explored in an extension of the tool using Maude's capability to communicate with \emph{external objects} (see \cite[Section 9]{}).