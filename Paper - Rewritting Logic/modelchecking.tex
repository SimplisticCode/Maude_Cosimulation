\section{Model checking Analysis of co-simulation}
This section describes the different model checking analysis we have performed using the framework.
The different analyses are based on the non-deterministic rewrites rule in Maude, which enable exploration of different instrumentations, algorithms and parameters of the same co-simulation scenario.

\subsection{Orchestration Algorithms}

\subsubsection{Verification to show that the orchestration is implementation}
To show that the a given orchestration algorithm is implementation-aware, meaning it satisfies all the contracts of the scenario.
We need to perform a symbolic co-simulation using the algorithm. The simulation is 
 where rewriting rules to 

\subsubsection{Deterministic }

\begin{lemma}[Deterministic Orchestration Algorithms]
    All implementation-aware orchestration algorithms of an instrumented scenario produce the same deterministic co-simulation result.
\end{lemma}

This is verified using the following search:
\begin{lstlisting}
    search (runAnyAlgorithm SCENARIO)  =>! S:SimulationState .
\end{lstlisting}

The search command relies heavily on rewriting rules, first generating the different orchestration algorithms for the instrumented scenario, which afterwards is executed either purely in Maude or using real FMUs.  

\begin{lstlisting}
crl [runAnyAlg] : runAnyAlgorithm INITSTATE => run: ORCHESTRATOR on: INITSTATE with: SIMULATIONDATA
if LOOPS := tarjan(INITSTATE)
    /\ SUIDsNE := getSUIDsOfScenario(INITSTATE)
    /\ SIMULATIONDATA := initialOrchestrationData(1,LOOPS,SUIDsNE)
    /\ ALGORITHMDATA := initialAlgorithmData(1)
    /\ CONF := calculateSNSet(INITSTATE ALGORITHMDATA) SIMULATIONDATA 
    /\ {CONF} => { FINALSTATE } 
    /\ ORCHESTRATOR := getOrchestrator(FINALSTATE)
    /\ allSUsinUnloaded(SUIDsNE, FINALSTATE) .
\end{lstlisting}

\subsection{Design Space Exploration}
Design space exploration is the .
This work explores two different kinds of design space exploration for co-simulation.


\subsubsection{Instrumentation of Scenario}
The beginning adaption of contracts for input ports to achieve more accurate co-simulation results is challenged by the lack of FMU-exporting tools not providing this information. 
To explore different instrumentations of a scenario, we have defined the contract: \textit{noContract}, that can be used if the user does not know the contract of the input.

In case the user wants to explore how a co-simulation scenario should be instrumented in order to obtain a desired co-simulation result.
The following rewrite rule should be used. 
\begin{lstlisting}
crl [findInstrumentation]: findContacts(INITSTATE) => CONF
    if findInstr(INITSTATE) => CONF
    /\ empty == tarjan(CONF) *** We do not allow loops in the found configuration
    /\ runAnyAlgorithm CONF => run: ORCHESTRATOR on: FINALSTATE with: SIMULATIONDATA
    /\ shouldSatisfy(FINALSTATE) .
\end{lstlisting}

The \textit{findInstr} tells Maude to explore all possible instrumentations of the scenario. 
In this example we place several constraints on the instrumentations we are interested in.
First, we do not allow instrumentations with algebraic loops. 
And finally we encode the constraints on the co-simulation result using the predicate \textit{shouldSatisfy}.

Using Maude's search command performs the search for instrumentations that leads to the desired co-simulation result.

\begin{lstlisting}
search findContacts(UninstrumentedScenario) =>! C:Configuration .
\end{lstlisting}

We believe that this is a natural and flexible way to explore how the different instrumentations affects the co-simulation result.
It is possible to restrict the explored instrumentations, such that they themselves satisfy specific properties for example no algebraic loops and to place criteria on the co-simulation result itself. 
The criteria on co-simulation result means that we can ensure that the instrumentation lead to a simulation result .


\subsubsection{Parameters of the Simulation Units}
A simulation unit may have different parameters affecting its behavior, so different parameters will lead to different simulation result.
Design space exploration is the process of exploring how the different parameters affects the co-simulation result adn find the optimal parameters. 

The Maude model allows to perform






We can potentially use SMT-solving later to help us with the design space exploration.

\subsection{Limitation of the approach}
We have performed the analyses on different scenarios. 
The state-space explosion is, of course, also applying to our work. 
However, we have been able to find the correct instrumentation and the contracts on systems with \simon{Create big scenario and test} SUs.

Our test examples include the industrial case study from Boeing presented in (Add reference to Gomes). 