% This is samplepaper.tex, a sample chapter demonstrating the
% LLNCS macro package for Springer Computer Science proceedings;
% Version 2.20 of 2017/10/04
%
\documentclass[runningheads]{llncs}

\usepackage{graphicx}
\usepackage{ae,aecompl}
\usepackage[utf8]{inputenc}
\usepackage[english]{babel}
\usepackage{verbatim}
\usepackage{graphicx}
\usepackage{amsfonts}
\usepackage{amsmath}
\usepackage{amssymb}
\usepackage{stmaryrd}
\usepackage{amstext}
\usepackage{bm} 
\let\proof\relax
\let\endproof\relax
\usepackage{amsthm}
\usepackage{siunitx}
\usepackage{mathrsfs}
\usepackage{wrapfig}
\usepackage{semantic}
\usepackage{algorithm}
\usepackage{algpseudocode}
\usepackage{float}
\usepackage{rotating}
\usepackage[dvipsnames]{xcolor}
\usepackage{paralist}
\usepackage[colorlinks]{hyperref}
\usepackage[capitalise,nameinlink]{cleveref}
\usepackage{url}
\usepackage[nounderscore]{syntax}
\usepackage{environ}
\usepackage{listings}
\usepackage{etoolbox}
\usepackage{subcaption}

\AtBeginEnvironment{algorithm}{\linespread{1.05}\selectfont}
\AtBeginEnvironment{grammar}{\small}
\AtBeginEnvironment{grammar}{\linespread{0.6}\selectfont}

% Comments

\newcommand{\simon}[1]{%
    {\scriptsize
        \textbf{\textcolor{red}{Simon: #1}}
    }%
}%

\newcommand{\jaco}[1]{%
    {\scriptsize
        \textbf{\textcolor{blue}{Jaco: #1}}
    }%
}%

\newcommand{\claudio}[1]{%
    {\scriptsize
        \textbf{\textcolor{magenta}{Claudio: #1}}
    }%
}%



% Paper specific notation

\setlength{\intextsep}{10pt}

\crefname{problem}{problem}{problems}

\newcommand{\inputV}{v}
\newcommand{\consistent}{\ensuremath{\mathit{Consistent}}}
\newcommand{\remaining}{\ensuremath{\mathit{Remaining}}}
\newcommand{\dontcare}{\_}
\newcommand{\defined}{\ensuremath{\mathit{defined}}}
\newcommand{\undefined}{\ensuremath{\mathit{undefined}}}
\newcommand{\properties}{P}
\newcommand{\satisfies}{\vDash}
\newcommand{\simulator}{\mathcal{A}}
\newcommand{\Induced}[2]{\llbracket #1 \rrbracket_{#2}}
\newcommand{\timebase}{\setreal_{\geq 0}}
\newcommand{\stepbase}{\setreal_{> 0}}
\newcommand{\backtrack}{B}

\newcommand{\stateset}[1]{S_{#1}}
\newcommand{\runstate}[1]{S^{R}_{#1}}
\newcommand{\state}[1]{s_{#1}}
\newcommand{\inputs}[1]{U_{#1}}
\newcommand{\inputvar}[1]{u_{#1}}
\newcommand{\outputs}[1]{Y_{#1}}
\newcommand{\outputvar}[1]{y_{#1}}
\newcommand{\mayReject}{\mathcal{M}}
\newcommand{\values}{\mathcal{V}}
\newcommand{\true}{\mathit{true}}
\newcommand{\false}{\mathit{false}}
\newcommand{\feedthrough}[1]{D_{#1}}
\newcommand{\reactivity}[1]{R_{#1}}
\newcommand{\fset}[1]{\mathtt{set}_{#1}}
\newcommand{\fget}[1]{\mathtt{get}_{#1}}
\newcommand{\fdoStep}[1]{\mathtt{doStep}_{#1}}
\newcommand{\fSave}[1]{\mathtt{Save}_{#1}}
\newcommand{\fRestore}[1]{\mathtt{Restore}_{#1}}
\newcommand{\maxStep}[1]{\mathtt{h}_{#1}}
\newcommand{\timestamp}[1]{\varphi(#1)}
\newcommand{\feedsto}[2]{U_{#1}^{#2}}
\newcommand{\CheckCon}{\mathit{CheckCon}}
\newcommand{\stepfound}{\mathit{Step\_found}}

\newcommand{\complexity}{Com}
\newcommand{\master}{\mathcal{A}}
\newcommand{\alloutputs}{Y}
\newcommand{\allfeedthroughs}{D}
\newcommand{\allreactivity}{R}
\newcommand{\allcontracts}{\mathcal{C}}
\newcommand{\coupling}{L}
\newcommand{\allinputs}{U}
\newcommand{\fmus}{C}
\newcommand{\fmusLoop}{C_{loop}}
\newcommand{\sequence}[1]{\pargroup{#1}}
\newcommand{\functioncall}{f}
\newcommand{\initcall}{I}
\newcommand{\allfunctioncalls}{F}
\newcommand{\fmu}[1]{\texttt{#1}}
\newcommand{\signal}[1]{\texttt{#1}}
\newcommand{\before}[2]{\ensuremath{#1 \twoheadrightarrow #2}}
\newcommand{\ibefore}[2]{\ensuremath{#1 \rightarrow #2}}
\newcommand{\after}[1]{{#1}'}
\newcommand{\aftern}[2]{{#1}^{(#2)}}
\newcommand{\stateafter}[2]{\ensuremath{\state{#1}^{(#2)}}}
\newcommand{\minstep}{\ensuremath{h_{min}}}

\algnewcommand\algorithmicforeach{\textbf{for each}}
\algdef{S}[FOR]{ForEach}[1]{\algorithmicforeach\ #1\ \algorithmicdo}
% New definitions
\algnewcommand\algorithmicswitch{\textbf{switch}}
\algnewcommand\algorithmiccase{\textbf{case}}
\algnewcommand\algorithmicassert{\texttt{assert}}
\algnewcommand\Assert[1]{\State \algorithmicassert(#1)}%
% New "environments"
\algdef{SE}[SWITCH]{Switch}{EndSwitch}[1]{\algorithmicswitch\ #1\ \algorithmicdo}{\algorithmicend\ \algorithmicswitch}%
\algdef{SE}[CASE]{Case}{EndCase}[1]{\algorithmiccase\ #1}{\algorithmicend\ \algorithmiccase}%
\algtext*{EndSwitch}%
\algtext*{EndCase}%






% Generic stuff

\newcommand{\footurl}[1]{\footnote{\url{#1}}}

% Math
\newcommand{\brackets}[1]{\ensuremath{ \left[ #1 \right] }}
\newcommand{\tuple}[1]{\ensuremath{ \left\langle #1 \right\rangle }}
\newcommand{\set}[1]{\ensuremath{ \left\{ #1 \right\}}}
\newcommand{\system}[1]{\ensuremath{ \begin{cases} #1 \end{cases}}}
\newcommand{\rightgroup}[1]{\ensuremath{ \left. \begin{matrix} #1 \end{matrix} \right\} } }
\newcommand{\pargroup}[1]{\ensuremath{ \left( #1 \right)}}
\newcommand{\inv}[1]{\ensuremath{\pargroup{ #1 }^{-1}}}
\newcommand{\dert}[1]{\ensuremath{ \dot{#1} }}
\newcommand{\ddert}[1]{\ensuremath{ \ddot{#1} }}
\newcommand{\partialder}[2]{\ensuremath{ \frac{\partial#1}{\partial#2} }}
\newcommand{\setreal}[0]{\ensuremath{\mathbb{R}}}
%\newcommand{\setbool}[0]{\ensuremath{\mathit{Bool}}}
\newcommand{\setnat}[0]{\ensuremath{\mathbb{N}}}
\newcommand{\norm}[1]{\left\lVert#1\right\rVert}
\newcommand{\bnorm}[1]{\big\lVert#1\big\rVert}
\newcommand{\abs}[1]{\left|#1\right\|}
\newcommand{\xs}[2]{\ensuremath{#1^{\left[#2\right]}}}
\newcommand{\infinitynorm}[1]{\left\lVert#1\right\rVert_\infty}


\newcommand{\vectorOne}[1]{\brackets{%
\begin{matrix}
  #1
 \end{matrix}%
}}
\newcommand{\vectorTwo}[2]{\brackets{%
\begin{matrix}
  #1 \\
  #2
 \end{matrix}%
}}
\newcommand{\vectorThree}[3]{\brackets{%
\begin{matrix}
  #1 \\
  #2 \\
  #3
 \end{matrix}%
}}
\newcommand{\vectorFour}[4]{\brackets{%
\begin{matrix}
  #1 \\
  #2 \\
  #3 \\
  #4
 \end{matrix}%
}}
\newcommand{\vectorFive}[5]{\brackets{%
\begin{matrix}
  #1 \\
  #2 \\
  #3 \\
  #4 \\
  #5
 \end{matrix}%
}}
\newcommand{\vectorSix}[6]{\brackets{%
\begin{matrix}
  #1 \\
  #2 \\
  #3 \\
  #4 \\
  #5 \\
  #6
 \end{matrix}%
}}
\newcommand{\vectorSeven}[7]{\brackets{%
\begin{matrix}
  #1 \\
  #2 \\
  #3 \\
  #4 \\
  #5 \\
  #6 \\
  #7
 \end{matrix}%
}}
\newcommand{\vectorEight}[8]{\brackets{%
\begin{matrix}
  #1 \\
  #2 \\
  #3 \\
  #4 \\
  #5 \\
  #6 \\
  #7 \\
  #8
 \end{matrix}%
}}

\newenvironment{aligneq*}%
{
\begin{equation*}
\begin{aligned}
}{
\end{aligned}
\end{equation*}
}

\newenvironment{aligneq}%
{
\begin{equation}
\begin{aligned}
}{
\end{aligned}
\end{equation}
}



%enable \cref{...} and \Cref{...} instead of \ref: Type of reference included in the link
%Nice formats for \cref
\usepackage{iflang}
\IfLanguageName{ngerman}{
  \crefname{table}{Tab.}{Tab.}
  \Crefname{table}{Tabelle}{Tabellen}
  \crefname{figure}{\figurename}{\figurename}
  \Crefname{figure}{Abbildungen}{Abbildungen}
  \crefname{equation}{Gleichung}{Gleichungen}
  \Crefname{equation}{Gleichung}{Gleichungen}
  \crefname{listing}{\lstlistingname}{\lstlistingname}
  \Crefname{listing}{Listing}{Listings}
  \crefname{section}{Abschnitt}{Abschnitte}
  \Crefname{section}{Abschnitt}{Abschnitte}
  \crefname{paragraph}{Abschnitt}{Abschnitte}
  \Crefname{paragraph}{Abschnitt}{Abschnitte}
  \crefname{subparagraph}{Abschnitt}{Abschnitte}
  \Crefname{subparagraph}{Abschnitt}{Abschnitte}
}{
  \crefname{section}{Sect.}{Sect.}
  \Crefname{section}{Section}{Sections}
  \crefname{listing}{\lstlistingname}{\lstlistingname}
  \Crefname{listing}{Listing}{Listings}
}
\crefname{section}{Sect.}{Sect.}
\Crefname{section}{Section}{Sections}
\crefname{listing}{\lstlistingname}{\lstlistingname}
\Crefname{listing}{Listing}{Listings}

\crefname{assumption}{Assumption}{Assumptions}




% Used for displaying a sample figure. If possible, figure files should
% be included in EPS format.
%
% If you use the hyperref package, please uncomment the following line
% to display URLs in blue roman font according to Springer's eBook style:
% \renewcommand\UrlFont{\color{blue}\rmfamily}

\begin{document}

\title{Co-Simulation is Rewriting Logic
\thanks{We are grateful to the Poul Due Jensen Foundation, which has supported the establishment of a new Centre for Digital Twin Technology at Aarhus University.}}

\author{Simon Thrane Hansen\inst{1} \orcidID{0000-0002-3796-4319} 
\and Peter Csaba Olveczky \inst{2} \orcidID{0000-0002-3796-4319}}
%
%\authorrunning{S. Thrane et al.}

\titlerunning{Co-Simulation using Rewriting Logic}

% First names are abbreviated in the running head.
% If there are more than two authors, 'et al.' is used.
%
\institute{
  DIGIT, Department of Electrical and Computer Engineering, Aarhus University, \\
  %\email{\{sth, claudio.gomes, casper.thule\}@ece.au.dk\} 
  \and Department of Computer Science, Oslo University - \textbf{Check this} \\
}

\maketitle              % typeset the header of the contribution
TACAS:\\
\textbf{Deadline: 14th of October}
\\
\textbf{Page limit: 16}\\
\textbf{Notification: 23rd of December}
\\
WRLA:\\
\textbf{Deadline: 28th of December}
\\
\textbf{Page limit: 15 (without references)}\\
\textbf{Notification: 8th of February}

Formal Framework for Synthesizing Co-simulation algorithms
, analyzing and executing co-simulations
\begin{itemize}
  \item Motivation for co-simulation
  \begin{itemize}
    \item Industry 4.0
    \item Development of CPS
    \item The result of co-simulation
    \begin{itemize}
    \item How to test and verify an OA - need for correct by construction algorithm
    \item Design Space Exploration of the system - model checking 
    \item Testing the configurations of the systems - model checking the contracts.
    \item Dynamic observation of the system
    \item Satisfying the contracts
    \item Checking safety properties (possible to reach a bad state). - model checking might be needed because although each SU might have deterministic behavior, we might not know this - we have to consider a range of possible behaviors.
    \item General problem with writing sound SU.
    \item Time 
  \end{itemize}
  \end{itemize}
  \item Background
  \begin{itemize}
    \item Rewriting Logic
    \item Co-simulation
    \begin{itemize}
      \item 
    \end{itemize}
  \end{itemize}
  \item The Maude model
  \begin{itemize}
    \item 
  \end{itemize}
  \item Analyses
  \begin{itemize}
    \item Confluent and deterministic algorithm
    \item Verification of Algorithms
    \item Design Space Exploration
    \item Step negotiation (Non-deterministic behavior)
    \item Contracts
    \item Interaction with real-world SUs
  \end{itemize}
\end{itemize}

Need for formal methods:
\begin{itemize}
  \item Desireable to do model checking as run-time verification during the simulation.
  \item Communication to external objects - continuous Interaction between formal tool.
  \item Step negotiation
  \item Fixed-point iteration (Algebraic loop)
  \item Complex functions - Tarjan, fixed-point iteration.
\end{itemize}

Contribution
\begin{itemize}
  \item Address the problems - obtaining good co-simulation algorithm
  \item Perform co-simulation in Maude
  \item Step negotiation
  \item Fixed-point iteration (Algebraic loop)
  \item Complex functions - Tarjan, fixed-point iteration.
\end{itemize}

Use of Maude:
\begin{itemize}
  \item Addresses the challenges:
  \item Non-distributed components
  \item Supports functions and is very compressive
  \item Supports model-checking - lets the user explore all different initialization of all the simulation units
  \item Model checking allows one to explore that the SU does not reach a bad state
  \item Most importantly, Maude supports communication with external objects so, in each simulation iteration, we can check and verify (model check) the algorithm used to simulate the next system step.
\end{itemize}

In this paper, we present a co-simulation Synthesizing engine and a Framework for performing various analyses in Maude of both the algorithm and the system of interest.
\begin{itemize}
  \item Model checking approach is used to explore the system's instrumentation's design space - check all possible contracts to reach the best simulation.
  This happens by 
  \item Given a state of the co-simulation system, we synthesize all correct orchestration (one of the algorithms might actually be better than another) for the next step of the co-simulation algorithm.
  \item Model checking is used to show that all synthesized algorithms lead to a  
  \item By defining abstract models of the system in Maude, we can perform model checking to explore violations of safety properties - to 
  \item We then show have this framework can be used to run a co-simulation - to let Maude use as the orchestration engine.
\end{itemize}

We illustrate our techniques with some examples.

Related work
\begin{itemize}
  \item Not a lot of formal methods on co-simulation
  \begin{itemize}
    \item All do work on a symbolic version of the simulation units
    \item Generate only one algorithm
  \end{itemize}
\end{itemize}

\section{Preliminaries}
\subsection{Co-simulation}
\subsection{Maude - Rewriting}
\subsubsection{Model checking}

\section{Formal model of co-simulation in Maude}
Abstract model in Maude 
\begin{itemize}
  \item Inputs
  \item Outputs
  \item Local state
  \item Parameters
\end{itemize}

\section{Synthesizing Algorithm}

\section{Model checking Analysis of co-simulation}
  \subsection{Orchestration Algorithms}
  \subsection{Design Space Exploration for instrumentation of the scenario}

\section{Use Maude an Orchestration Engine}

Interaction with real system

\section{Case study}

\section{Related Work}

\section{Concluding remarks}



TODAY:
\begin{itemize}
  \item Define the Algorithm Verification
  \begin{itemize}
    \item Introduce non-deterministic SUs
  \end{itemize}
  \item Perform various analyses
  \begin{itemize}
    \item Design Space Exploration - instrumentation
    \item Checking Safety Properties
    \item Design Space of the system
  \end{itemize}
  \item Make equation into rules.
\end{itemize}

External objects in Aarhus - ask Casper -maybe implement something in Python:



\section{Progress on Thursday day}
\begin{itemize}
  \item Simulation one synthesized algorithm
  \item Generation of all algorithms - multiple
  \item Show that all algorithms are confluent - one final state
  \item Reachability analyses - can we reach any bad state with any algorithm.
  \item Run for one algorithm - small step semantics
  \item Design Space Exploration:
  \begin{itemize}
    \item Instrumentation of the ports - with no loops and that satisfied certain properties - (show casing two different ways)
    \item Finding the different parameters of the simulation units from a discrete set
    \item Combination of the two.
  \end{itemize}
  \item  Big step simulation running multiple iterations
\end{itemize}

\section{Todo:}

\begin{itemize}
  \item Simulate the system with loops
  \item Rules vs. Equations
  \item External objects - long run
  \item Tidying up
  \item Larger example - to show the different things
\end{itemize}

%\begin{abstract}
Simulation-based analyses of cyber-physical systems are increasingly vital in the era of Industry 4.0. 
Co-simulation enables the coupling of specialized simulation tools through an orchestration algorithm. 
The orchestrator dictates how each simulation tool should simulate its corresponding subsystem.
Obtaining correct simulation results requires an implementation-aware orchestration algorithm tailored to the specific scenario, without the orchestrator knowing each simulation tool's implementation. 
The orchestrator should stabilize algebraic loops, perform time step negotiation, and adhere to each simulation tool's implementation.
This paper describes an approach and implementation for generating orchestration algorithms that respect all simulation units' implementation contracts by construction using rewriting logic. 
Furthermore, the approach enables various analyses of the orchestration algorithm and system itself, among other various kinds of design space exploration.
     
\keywords{Co-Simulation \and Rewriting Logic \and Model-checking \and Cyber-Physical Systems}
\end{abstract}





%% % Motivation for co-simulation
% Cyber-physical systems (CPSs) are omnipresent and part of the critical infrastructure.
% A CPS is a hybrid system that embodies physical processes controlled
% by digital devices. Examples include everything from a controller
% opening and closing a valve to regulate the water level in a tank to
% autonomous vehicles. 
% CPSs are becoming increasingly complex and critical~\cite{4519604},
% which leads to the desire for techniques to assist in their
% development. 
% Traditional modeling and simulation techniques, where a solver
% algorithm describes a model, are no longer sufficient to cope with
% the integrated development processes of such systems~\cite{FMI2014},
% which consists of heterogeneous subsystems typically developed using
% different tools and formalisms  
%
% % Why and what is co-simulation
% %Co-simulation is the study of how to coordinate multiple black-box
% simulation units (SUs) developed using different tools and
% formalisms. 
% Co-simulation solves the shortcomings of the traditional modeling
% techniques by letting the user explore the behavior of a system
% described as a composition of black-box subsystems
% (SUs)~\cite{Kubler2000,Gomes2018}. 
%
% An SU implements a well-defined interface and represents a subsystem
% by calculating its behavioral trace using a dedicated solver. 
% An example of such an SU is a Functional Mock-up Unit (FMU) defined
% by the Functional Mock-up Interface Standard~\cite{FMI2014} (FMI),
% which inspires the notion of an SU in this paper.  
% FMI is widely adopted, supported by many tools~\cite{Tools_FMI}, and
% used commercially. 
%
% %How to test and verify an OA - need for correct by construction algorithm
% The overarching challenge of co-simulation is ensuring accurate
% simulation results due to the many potential error
% sources~\cite{Gomes2018}. 
% The simulation result depends not only on the correctness of the
% individual SUs but also on the co-simulation algorithm that composes
% them, which is often
% overlooked~\cite{Gomes2019,Gomes2018f,Schweizer2015c}. 
% The co-simulation algorithm must be specifically tailored to the
% instrumentation of the scenario to ensure an accurate simulation
% result~\cite{hansen_verification_2021,Oakes2021}. 
% Not considering the scenario's instrumentation can lead to
% inaccurate co-simulation results, which can be extremely difficult
% to debug and fix in practice. 
% The challenge has partly been addressed by the papers
% \cite{Gomes2019,Oakes2021,thrane2021} that show how to synthesize
% tailored co-simulation algorithms. 
% Nevertheless, the approaches do not consider scenarios where the
% instrumentation is unknown or the SUs are parametric. 
%
% %Design space evaluation
% Co-simulation creates an environment where SUs can be composed to
% explore and analyze different system's design to find the optimal
% design~\cite{dse,gamble_design_2014}; the process is called design
% space exploration. 
%
% We propose a framework in Maude where a co-simulation practitioner
% can synthesize  and execute tailored co-simulation algorithms while
% exploring  different system designs for both simple and complex scenarios.
% Complex scenarios are subject to algebraic loops denoting cyclic dependencies between the SUs or step rejections, where an SU refuses specific future state evaluations because it implements error estimation.
% Complex scenarios are more challenging to simulate since the co-simulation algorithm needs to adapt to the behavior of the SUs~\cite{thrane2021} to solve algebraic loops and ensure that all SUs move in lockstep.
%
% Our framework enables a co-simulation practitioner to explore the consequences of changing the scenario's instrumentation and SU parameters. 
% Co-simulation practitioners can thereby use their domain knowledge to place constraints on the co-simulation result to let Maude find the parameters and instrumentation that result in a co-simulation with desirable properties. 
%
%
% =============================

\section{Introduction}

Modern  cyber-physical systems (CPSs),  such as,  e.g.,  nuclear power
plants, cars, and airplanes, consist of multiple
heterogeneous subsystems that    are developed by different
companies using different tools and formalisms~\cite{Paris19}.
These companies will usually not share their models for commercial reasons.
There is nevertheless  a need to determine   how
these subsystems 
interact and to  explore and analyze different design choices as early
as possible~\cite{4519604}. 
One way of addressing this need is to use, for each subsystem,  an
interface that provides an abstraction of that subsystem. 
\emph{Simulation
units} (SUs) provide such abstractions and are widely used in
industry. A class  of SUs are   described by
the Functional  Mock-up Interface Standard~\cite{FMI2014} (FMI), which
is used commercially and is  supported by many
tools~\cite{Tools_FMI}. 
An SU implements a well-defined interface and represents a subsystem
by computing   its behavioral trace using a dedicated solver.

\emph{Co-simulation}~\cite{Kubler2000,Gomes2018} addresses the need to
simulate a  CPS given as the
composition of  such black-box SUs.
Co-simulation transforms a continuous system to a discrete simulation
with discrete  interactions between the different SUs.
Furthermore, a \emph{digital twin} can be a co-simulation connected to
a  physical  systems.

The   objective of a
co-simulation is to capture as accurately as possible   
the behavior of  the modeled system. This is challenging,  due
to discretization,  cyclic dependencies between the SUs,  and the fact
that very few  assumptions be made about the SUs: an SU may, e.g., be
unable to predict  its future state at the next desired point in time.
A \emph{co-simulation algorithm} is responsible for orchestrating the
interaction of the SUs:   it determines how and when the different SUs
interact. 

Since the co-simulation algorithm should make   the virtual
system correspond  to its physical
counterpart,   the virtual system can be analyzed,
and different design  choices can be explored,  to accurately predict
the behavior of the  system to be built. 
%  
However, the FMI standard is only informally described, and has been
shown to be inconsistent \cite{sampaio_behavioural_2016}.  For both of
the above reasons,  there is
 a need for 
formal methods to provide a formal semantics for co-simulation and to
provide early model-based formal analysis of the co-simulations.   

However, providing a formal semantics to co-simulation is challenging,
due to, e.g.,  the
complex behavior of  the SUs, and the need to resolve cyclic
dependencies between 
the SUs by fixed-point computations and to  perform step negotiation
to ensure that all SUs move in  lockstep.
Rewriting logic~\cite{Mes92}, with its  modeling language and
high-performance  analysis tool Maude~\cite{maude-book}, 
 should be a suitable  formal method for co-simulation: 
 Its expressiveness allows us to conveniently specify both  
 complex dynamic behaviors and 
sophisticated  functions  (e.g., for detecting and resolving cyclic
dependencies), and 
Maude  provides  automatic  formal analysis capabilities for
correctness analysis and 
design space  exploration.
Maude also supports connections to \emph{external objects},
which means  that  Maude   should be able to 
 orchestrate  the composition of real  external components. 

In this paper we present a formal framework for representing
co-simulation in Maude. 
We give a formal model for co-simulation beyond the FMI 2.0 standard, 
also covering  feed-through constraints, input instrumentations, and
step rejection.  
We then use Maude to synthesize and execute suitable
(scenario-specific) 
co-simulation  algorithms, which enables the formal analysis of the
resulting  co-simulation. 
We also show how Maude can be used to synthesize
instrumentations,  parameter values,  and co-simulation algorithms for
such complex  scenarios so that the resulting system satisfies
desired properties. 
% 
As discussed in \Cref{sc:related}, to the best of our
knowledge this paper presents  the first formal
framework that covers  design space exploration of complex
co-simulation scenarios  with algebraic loops and step
rejection,  and that also  synthesizes correct-by-construction
co-simulation algorithms and  parameters for  such  scenarios.

From a Maude perspective, we found that using rewrite conditions in
 rules  allowed us to  easily and
elegantly  solve quite   challenging problems in co-simulation.


The rest of the paper is structured as follows.
\Cref{sc:background} provides  necessary background to Maude and
co-simulation. 
\Cref{sec:model} presents a Maude model of co-simulation
scenarios and SU  behaviors.
\Cref{sc:synthesize} shows how correct-by-construction co-simulation
algorithms can be  synthesized and executed in Maude.
\Cref{sc:DSE} describes how to synthesize instrumentation and
parameter values such  that resulting co-simulation satisfies desired
properties. 
\Cref{sc:related} discusses related work and \Cref{sc:summary} gives some
concluding  remarks.

%\section{Preliminaries}\label{sc:background}
This section describes the preliminaries to understand the work described in the paper.

\subsection{Maude /Rewriting Logic}
\simon{Peter, I guess you have some text for this section.}

\subsubsection{Model checking}
\simon{Is this section needed?}


\subsection{Co-simulation}
Co-simulation enables global simulation of a system consisting of multiple black-box SUs. 
An SU has a dedicated solver calculating the behaviour trace of the dynamical system it represents. 
A dynamical system represents a function from time and space into some often multi-dimensional and continuous space. 
Examples include population growth, water flow, and pendulums. 
Interaction of the system with the external environment happens through inputs and outputs~\cite{Gomes2019a,Kubler2000}.

SUs are coupled through their inputs and outputs.
A coupling indicates that the state of one SU is reliant on the state of another SU.
A coupling can be seen as an invariant saying that the value on a coupled input and output must be identical at all times - known as a coupling restriction. 
However, in practice, the coupling restrictions can only be satisfied at specific points in time, referred to as communication points. 
Therefore, each SU makes assumptions about the evolution of the input values between the communication points to satisfy the invariant.
These assumptions cause accumulable errors in the co-simulation~\cite{Arnold2014}.

The orchestrator computes the behavioural trace of all SUs and tries to satisfy their coupling restrictions by exchanging values. 
The orchestrator aims to find the communication points minimizing the co-simulation error while ensuring that the SUs move in lockstep. 
Studies \cite{Gomes2019,Oakes2021,Gomes2018f,Schweizer2015c,Gomes2018a} show that optimal communication points depend on the implementation of the SUs.

We now introduce our definition of an SU in \cref{def:fmu}.
The definition is based on \cite{Broman2013,Gomes2019c,thrane2021} and represents an abstract version of an SU.

\begin{definition}[Simulation Unit]\label{def:fmu}
  An SU with identifier $c$ is represented by the tuple
  $$\tuple{\stateset{c}, \inputs{c}, \outputs{c}, \fset{c}, \fget{c}, \fdoStep{c}},$$
  where:
  \begin{compactitem}
    \item $\stateset{c}$ represents the state space.
    \item $\inputs{c}$ and $\outputs{c}$ the set of input and output variables, respectively.
    The union of the inputs and outputs is called the variables of the SU:
    $\variables{c} = \inputs{c} \cup \outputs{c}$ 
    \item $\values$ is the set of values that a variable can hold and let $\valuesExchanged: \timebase \times \values$ be the set of abstract values exchanged between input and output variables.
    The functions
    $\fset{c} : \stateset{c} \times \inputs{c} \times \valuesExchanged \to \stateset{c}$ and $\fget{c}: \stateset{c} \times \outputs{c} \to \valuesExchanged$ respectively sets an inputs and gets an outputs. 
    \item $\fdoStep{c}: \stateset{c} \times \stepbase \to \stateset{c} \times \stepbase $ is a function that instructs the SU to compute its state after a given time duration. If an SU is in state $\stateafter{c}{t}$ at time $t$, $(\stateafter{c}{t+h}, h) = \fdoStep{c}(\stateafter{c}{t}, H)$ approximates the state $\stateafter{c}{t+h}$ of the corresponding model at time $t+h$, where $h \leq H$. 
  \end{compactitem}
\end{definition}

The state of SU $A$ at time $t$ is denoted $\stateafter{A}{t}$.
The function $\fdoStep{c}$ returns a step size because some SUs implement error estimation and may conclude that taking a step size of $H$ will result in an intolerable error meaning the SU takes a smaller step than planned.

A collection of coupled SUs forms a scenario.
We provide a formal description of a scenario in \cref{def:cosim_scenario}.

\begin{definition}[Scenario]\label{def:cosim_scenario}
  A scenario is a structure $\tuple{\fmus, \coupling, \mayReject, \allfeedthroughs, \allreactivity, \alldelayed}$ where 
  \begin{compactitem}
    \item Each identifier $c \in \fmus$ is associated with an SU, as defined in \cref{def:fmu}
    \item $\coupling(u)=y$ means that the output $y$ is connected to input $u$.   
    Let $\allinputs = \bigcup_{c \in \fmus} \inputs{c}$ and $\alloutputs = \bigcup_{c \in \fmus} \outputs{c}$, then $\coupling : \allinputs \to \alloutputs$. 
    \item $\mayReject \subseteq \fmus$ denotes the SUs that implement error estimation. 
    \item The function
    $\allreactivity : \allinputs \to \mathbb{B}$ is total and provide non-confidential information about the SUs' input approximation functions.
    $\allreactivity(\inputvar{c}) = \true$ means that the function $\fdoStep{c}$ assumes that the timestamp $t_{SU}$ of the state of SU $c$ ($\stateafter{c}{t_{SU}}$) is less than the timestamp $t_v$ of the value $v = \tuple{t_V,\dontcare}$ set on the input $\inputvar{c}$.
    \item  Finally, the set of feed-through components, $\allfeedthroughs = \bigcup_{c \in \fmus} \feedthrough{c}$, where the input $\inputvar{c} \in \inputs{c}$ feeds through to output $\outputvar{c} \in \outputs{c}$, that is, $(\inputvar{c},\outputvar{c}) \in \feedthrough{c}$, when there exists $v_1, v_2 \in \valuesExchanged$ and $\state{c} \in \stateset{c}$, such that
    $\fget{c} (\fset{c}(\state{c}, \inputvar{c}, v_1), \outputvar{c}) \neq \fget{c} (\fset{c}(\state{c}, \inputvar{c}, v_2), \outputvar{c}).$
  \end{compactitem}
\end{definition}

We have based \cref{def:cosim_scenario} on the FMI standard. 
However, feed-through and reactivity are extensions of the standard, introduced to cover a broad class of co-simulation scenarios.
We use the syntax in \cref{fig:simpleexample} to graphically present co-simulation scenarios.

\begin{figure}[htb]
  \centering
  \includegraphics[width=0.7\textwidth]{images/simple_example.pdf}
  \caption{A co-simulation scenario ($S1$).}
  \label{fig:simpleexample}  
\end{figure}

\subsubsection{Instrumentation}
We refer to an input variable $\inputvar{c} \in \allinputs \land \neg \allreactivity(\inputvar{c})$ as a delayed input. 
While an input variable $\inputvar{c}$ where $\inputvar{c} \in \allinputs \land \allreactivity(\inputvar{c})$ is referred to as a reactive input. 

The function $\allreactivity$ is the instrumentation of the scenario.
The instrumentation describes the input approximation functions of the different SUs.
%An example of the instrumentation function
Changing the instrumentation of a scenario changes the algorithm used to simulate the scenario.
We assume that the instrumentation of a scenario is constant through the simulation, which is the case for most commercially used SUs.

\subsection{Co-simulation algorithms}\label{sc:cosimalgo}
The orchestrator simulates the scenario by interpreting/executing a co-simulation algorithm on the scenario.

A co-simulation algorithm consists of an initialization procedure, a co-simulation step, and some additional FMI functions changing the mode of an SU~\cite{FMI2014}.
Our work concentrates in the paper on the co-simulation step, which we refer to as the algorithm throughout the paper. 
The reason is that the other aspects of a co-simulation algorithm can be derived from the presented method.  

The algorithm changes the co-simulation state. 
Our work is performed using an abstract co-simulation state, which is defined as the combination of the state of the individual SUs in \cref{def:cosimstate}.

\Cref{def:runtime_state} defines the abstract state of an SU.

\begin{definition}[Abstract State]\label{def:runtime_state}
  Given an SU $c$ as defined in \cref{def:fmu}, the observable abstract state of $c$ is a member of the set $\runstate{c} = \timebase \times \runstate{\inputs{c}} \times \runstate{\outputs{c}} \times \runstate{V_c}$, where:
  \begin{compactitem}
    \item $\timebase$ is the time of the SU meaning the state of the SU is $\stateafter{c}{t}$,
    \item $\runstate{\inputs{c}} : \inputs{c} \to \timebase$ is a total function linking each input port with its timestamp.  
    \item $\runstate{\outputs{c}} : \outputs{c} \to \timebase$ is a total function linking each output port with its timestamp.  
    \item $\runstate{V_c} : \variables{c} \to \values$ is a total function linking each port with a its value.  
  \end{compactitem}
\end{definition}

We use the abstract state $\runstate{c}$ of an SU $c$ instead of the internal state $\stateset{c}$ defined in \cref{def:fmu} because the orchestrator cannot observe the later.

\begin{definition}[Abstract Co-simulation State]\label{def:cosimstate}
  Given a co-simulation scenario $\tuple{\fmus, \coupling}$, as defined in \cref{def:cosim_scenario}, the abstract co-simulation state is a member of the set $\runstate{\fmus} = \prod_{c \in \fmus} \runstate{c}$. 
\end{definition}

A co-simulation step $P$ is a sequence of the functions $\fset{c},\fget{c}$, and $\fdoStep{c}$ described in \cref{def:fmu}.
It simulates the scenario by advancing all SUs $\fmus$ from an initial state at time $t$ to a final state at time $t+H, \textrm{ where } H > 0$.
The co-simulation must ensure that the coupling restrictions are satisfied at both the initial and final state.

\Cref{fig:algorithms} shows three different co-simulation steps of the scenario in \cref{fig:simpleexample}.
The three different algorithms are all allowed by the FMI standard~\cite{FMI2014}. 

\begin{figure}[htb]
  \centering
  \begin{minipage}[t]{.325\textwidth}
    \begin{algorithm}[H]
      \caption{}
      \label{alg:algorithm_1}
      \begin{algorithmic}[1]
        \scriptsize
        \State $(\stateafter{A}{H},H) \gets \fdoStep{A}(\stateafter{A}{0}, H)$
        \State $(\stateafter{B}{H},H) \gets \fdoStep{B}(\stateafter{B}{0}, H)$
        \State $f_{v} \gets \fget{A}(\stateafter{A}{H}, \outputvar{f})$
        \State $g_{v} \gets \fget{B}(\stateafter{B}{H}, \outputvar{g})$
        \State $\stateafter{B}{H} \gets \fset{B}(\stateafter{B}{s}, \inputvar{f}, f_{v})$
        \State $\stateafter{A}{H} \gets \fset{A}(\stateafter{A}{H},\inputvar{g},g_{v})$
      \end{algorithmic}
    \end{algorithm}
  \end{minipage}
  \begin{minipage}[t]{0.325\textwidth}
    \begin{algorithm}[H]
      \caption{}
      \label{alg:algorithm_2}
      \begin{algorithmic}[1]
        \scriptsize
        \State $(\stateafter{B}{H},H) \gets \fdoStep{B}(\stateafter{B}{0}, H)$
        \State $(\stateafter{A}{H},H) \gets \fdoStep{A}(\stateafter{A}{0}, H)$
        \State $g_v \gets \fget{B}(\stateafter{B}{H}, \outputvar{g})$
        \State $\stateafter{A}{H} \gets \fset{A}(\stateafter{A}{H}, \inputvar{g}, g_v)$
        \State $f_v \gets \fget{A}(\stateafter{A}{H}, \outputvar{f})$
        \State $\stateafter{B}{H} \gets \fset{B}(\stateafter{B}{H}, \inputvar{f}, f_v)$
      \end{algorithmic}
    \end{algorithm}
  \end{minipage}
  \begin{minipage}[t]{0.325\textwidth}
    \begin{algorithm}[H]
      \caption{}
      \label{alg:algorithm_3}
      \begin{algorithmic}[1]
        \scriptsize
        \State $(\stateafter{B}{H},H) \gets \fdoStep{B}(\stateafter{B}{0}, H)$
        \State $g_v \gets \fget{B}(\stateafter{B}{H}, \outputvar{g})$
        \State $\stateafter{A}{0} \gets \fset{A}(\stateafter{A}{0}, \inputvar{g}, g_v)$
        \State $f_v \gets \fget{A}(\stateafter{A}{0}, \outputvar{f})$
        \State $\stateafter{B}{H} \gets \fset{B}(\stateafter{B}{H}, \inputvar{f}, f_v)$
        \State $(\stateafter{A}{H},H) \gets \fdoStep{A}(\stateafter{A}{0}, H)$
      \end{algorithmic}
    \end{algorithm}
    \vspace{4pt}
  \end{minipage}
  \vspace{-2em}
  \caption{Three algorithms conforming to the FMI Standard (version 2.0) of the scenario in \cref{fig:simpleexample}.}
  \label{fig:algorithms}
\end{figure}

Although the three algorithms in \cref{fig:algorithms} consist of the same actions, they are not equivalent, and simulating with one algorithm instead of one of the others could drastically change the co-simulation result as shown in \cite{Gomes2019c,hansen_verification_2021}. 
At the end of \cref{sec:correctcosim}, we show which of these algorithms is correct.

The purpose of the co-simulation step $P$ is defined in \cref{def:comsim_step}.

\begin{definition}[Co-simulation Step]\label{def:comsim_step}
  A co-simulation step $P$ is correct starting from the abstract state $\runstate{}$ if:
  \begin{align}
    &\fpreCoSimStep(\tuple{t,\runstate{\allinputs}, \runstate{\alloutputs},\runstate{V}}, t) \triangleq 
    \forall \inputvar{c} \in \allinputs
    \exists \outputvar{d} \in \alloutputs
    \cdot \coupling(\inputvar{c}) = \outputvar{d}  \nonumber \\
    &\qquad \qquad \qquad \qquad \qquad \implies
    \runstate{V}(\inputvar{c}) = \runstate{V}(\outputvar{d}) \nonumber \\
    &[\fpreCoSimStep(\runstate{}, t), 
    \fpreCoSimStep(\after{\runstate{}}, t+H)] 
    \langle \after{\runstate{}} \gets P\rangle
  \end{align}
\end{definition}

$\runstate{} = \tuple{t, \runstate{\allinputs}, \runstate{\alloutputs}, \runstate{V}}$ means that all SUs are at time $t$ and $\runstate{\allinputs} = \prod_{c \in \fmus}\runstate{\inputs{c}}$ and $\runstate{\alloutputs} = \prod_{c \in \fmus}\runstate{\outputs{c}}$.
\Cref{def:comsim_step} shows that the precondition and postcondition of the co-simulation step are the same ($\fpreCoSimStep$).
It says that all SUs are synchronized and that all the coupling restrictions are satisfied.

The three Algorithms in \cref{fig:algorithms} all satisfy the criteria of \label{eq:co_sim_step} even though they can result in utterly different simulation results.
To discriminate between them, we need to look into the semantics of the different actions from \cref{def:fmu}, which we describe next in \cref{def:getout,def:setin,def:step}.

We base our semantic on \cite{Gomes2019a,hansen_verification_2021}.
\simon{Make reference to }
Due to space limitations we limitations we refer readers to these papers for a deeper explanation of the actions.

\begin{definition}[Get Action]\label{def:getout}  
  For an SU $c$ at timestamp $t$ with the abstract state $\runstate{c} = \tuple{t,\runstate{\inputs{c}}, \runstate{\outputs{c}}, \runstate{V_c}}$ the effect of obtaining a value from an output $\outputvar{c}$ using the action $\fget{c}(\stateafter{c}{t},\outputvar{c})$ is described using the specification statement:
  \begin{align}
    &\fpreget{c}(\outputvar{c}, \tuple{t,\runstate{\inputs{c}}, \runstate{\outputs{c}}, \runstate{V_c}}) \triangleq
    \runstate{\outputs{c}}(\outputvar{c}) < t \land \nonumber \\
    & \qquad \qquad \qquad \qquad \qquad 
    \forall \inputvar{c} \in \inputs{c} \cdot (\inputvar{c}, \outputvar{c}) \in \feedthrough{c} 
    \implies \runstate{\inputs{c}}(\inputvar{c}) = t  \nonumber \\
    &\fpostget{c}(\outputvar{c}, \tuple{t,\runstate{\inputs{c}}, \runstate{\outputs{c}}}, \tuple{t,\runstate{\inputs{c}}, \after{\runstate{\outputs{c}}}}, v) \triangleq 
    \after{\runstate{\outputs{c}}}(\outputvar{c}) = t \nonumber \\
    & \qquad \qquad \qquad \qquad \qquad 
    \land 
    \forall \outputvar{m} \in (\outputs{c} \setminus \outputvar{c}) \cdot 
    \after{\runstate{\outputs{c}}}(\outputvar{m}) =
    \runstate{\outputs{c}}(\outputvar{m})\nonumber\\
    &\qquad \qquad \qquad \qquad \qquad \land
    \after{\runstate{V_c}}(\outputvar{c})= v \nonumber\\
    &[\fpreget{c}(\outputvar{c}, \runstate{c}), 
    \fpostget{c}(\outputvar{c}, \runstate{c}, \after{\runstate{c}}, v)] 
    \langle (v, \after{\runstate{c}}) \gets \fget{c}(\stateafter{c}{t},\outputvar{c}) \rangle \nonumber
  \end{align}
\end{definition}

\begin{definition}[Set Action]\label{def:setin}
  For an SU $c$ at timestamp $t$ with the abstract state $\runstate{c} = \tuple{t,\runstate{\inputs{c}}, \runstate{\outputs{c}}, \runstate{V_c}}$ the effect of 
  setting a value $\inputV = \tuple{t_{V}, X}$ on the input $\inputvar{c}$ using the action $\fset{c}(\stateafter{c}{t}, \inputvar{c}, \inputV)$ is described using the specification statement:
    \begin{align}
      &\fpreset{c}(\inputvar{c}, \tuple{t_{V},X}, \tuple{t,\runstate{\inputs{c}}, \runstate{\outputs{c}}, \runstate{V_c}}) \triangleq 
      let\; t_s = \runstate{\inputs{c}}(\inputvar{c}) \; in \; t_s < t_{V} \nonumber \\
      & \qquad \qquad \qquad \qquad \qquad
      \land 
      ((\reactivity{c}(\inputvar{c}) \land t_s = t) 
      \lor (\neg\reactivity{c}(\inputvar{c}) \land t_s < t)) \nonumber \\
      &\fpostset{c}(\inputvar{c}, \tuple{t_{V},X}, \tuple{t,\runstate{\inputs{c}}, \runstate{\outputs{c}}, \runstate{V_c}}, 
      \tuple{t, \after{\runstate{\inputs{c}}} \runstate{\outputs{c}}, \after{\runstate{V_c}}}) \triangleq 
      t_{V} = \after{\runstate{\inputs{c}}}(\inputvar{c})
      \nonumber\\
      &\qquad \qquad \qquad \qquad \qquad \land
      \forall \inputvar{m} \in (\inputs{c} \setminus \inputvar{c}) \cdot 
      \after{\runstate{\inputs{c}}}(\inputvar{m}) =
      \runstate{\inputs{c}}(\inputvar{m}) 
      \nonumber \\
      &\qquad \qquad \qquad \qquad \qquad \land
      \after{\runstate{V_c}}(\inputvar{c}) = X \nonumber\\
      &[\fpreset{c}(\inputvar{c}, \runstate{c}), 
      \fpostset{c}(\inputvar{c}, \inputV, \runstate{c}, \after{\runstate{c}})] 
      \langle \after{\runstate{c}} \gets \fset{c}(\stateafter{c}{t},\inputvar{c}, \inputV) \rangle \nonumber
    \end{align}
  \end{definition}


  \begin{definition}[Step Computation]\label{def:step}
    For an SU $c$ at timestamp $t_{SU}$ with the abstract state $\runstate{c} = \tuple{t_{SU},\runstate{\inputs{c}}, \runstate{\outputs{c}, \runstate{V_c}}}$ the effect of stepping it with the step duration $H$ using the action $\fdoStep{c}(\stateafter{c}{t}, H)$ is defined using the following specification statement:
    \begin{align}
      &\fpredoStep{c}(H, \tuple{t,\runstate{\inputs{c}}, \runstate{\outputs{c}, \runstate{V_c}}}) \triangleq 
      \forall \inputvar{c} \in \inputs{c}
      \cdot 
      ((\reactivity{c}(\inputvar{c}) \land t_{SU} + H = \runstate{\inputs{c}}(\inputvar{c}))
      \nonumber \\
      &\qquad \qquad \qquad \qquad \qquad \qquad \qquad \qquad \qquad 
      \lor 
      (\neg \reactivity{c}(\inputvar{c}) \land t_{SU} = \runstate{\inputs{c}}(\inputvar{c})))
      \nonumber \\
      &\fpostdoStep{c}(H, \tuple{t,\runstate{\inputs{c}}, \runstate{\outputs{c}}, \runstate{V_c}}, \tuple{t',\runstate{\inputs{c}}, \runstate{\outputs{c}}, \after{\runstate{V_c}}}) \triangleq t + H = t' \nonumber \\
      &[\fpredoStep{c}(H, \runstate{c}), 
      \fpostdoStep{c}(H, \runstate{c}, \after{\runstate{c}})] 
      \langle (\after{\runstate{c}}, H) \gets \fdoStep{c}(\stateafter{c}{t},H) \rangle \nonumber
    \end{align}
  \end{definition}

%   \cref{def:step} describes that to advance the state of an SU, all its inputs must have been correctly updated with respect to their contract since the last $\fdoStep{c}$ action. 
%   By updating the inputs, we mean that their inputs have been properly advanced in time by the $\fset{c}$ action defined in \cref{def:setin}.   
%   This is informally what $\fpredoStep{c}$ says.

%   The effect of stepping an SU is that the SU's abstract state gets advanced with the duration $H$ this is defined by $\fpostdoStep{c}$.
%   Note that we, for now, assume that an SU will accept all step duration. 
%   Step rejection is described in \vref{sc:complex}.

%   \cref{def:step,def:setin,def:getout} describe the semantics of the actions, both their precondition and effect on the co-simulation state.



\subsection{Complex Scenarios}
Complex scenarios are subject to algebraic loops or step rejections.
A complex scenario is simulated using an iterative orchestration algorithm.
\simon{Add scenario /figure}

The algorithm is iterative because the orchestrator needs to adapt to the behavior of the SUs to satisfy the constraints associated with each SU, account for possible step rejections, and solve algebraic loops. 
The orchestrator achieves this by finding a correct valuation of all inputs and outputs in the scenario.
The valuation defines the step duration and the values it uses to solve the scenario's algebraic loops. 
A correct valuation ensures that all SUs agree on the step duration and solves all algebraic loops by finding a fixed point.
A co-simulation can only be correctly simulated using a correct valuation.

The orchestrator must find a correct valuation in every co-simulation step because it depends on the current internal state of the SUs, which varies from one co-simulation step to the next.

Due to space limitations, we will not describe the process for finding a correct valuation. 
Interested readers are referred to \cite{thrane2021}.

\subsection{Correct Co-simulation Algorithms}\label{sec:correctcosim}
To optimally simulate a co-simulation scenario using an algorithm $P$ requires more than a correct valuation $v$. 
An algorithm $P$ is correct if it satisfies \cref{def:comsim_step} and that $P$ changes the co-simulation state of the scenario such that only enabled actions are performed.

We can now describe what it means for an algorithm to be correct for a given scenario in the following Hoare triple.

\begin{definition}\label{def:correctalgo}
  An algorithm $P$ and valuation $v: \tuple{H, \dontcare}$ is correct if:
  \begin{align*}
     \quad \land \quad
     &\{\forall v \in \allinputs \cup \alloutputs \ldotp v = \tuple{t, \dontcare} \land \forall c \in \fmus \ldotp \ftime(\stateafter{c}{t}) = t\} \quad P(c) \\
     & \{\forall v \in \allinputs \cup \alloutputs \ldotp v = \tuple{t+H, \dontcare} \land \forall c \in \fmus \ldotp \ftime(\stateafter{c}{t+H}) = t+H \}
  \end{align*}
  Meaning all preconditions are satisfied and all SUs and inputs have moved from time $t$ to time $t+H$ through the execution of $P(c)$.
\end{definition}

We use \cref{def:correctalgo} to conclude that \cref{alg:algorithm_3} is correct while the others are incorrect since they break one or more of the defined preconditions. 
\Cref{alg:algorithm_1,alg:algorithm_2} violate the precondition of $\fdoStep{a}$ on line 2 by stepping it without having provided SU $a$ with a value on the reactive input $g$. 
These definitions form the basis for describing the approach for synthesizing and verifying co-simulation algorithms in this work.

\subsection{Design Space Exploration}
Design space exploration is a technique for evaluating how different combinations of parameters (designs) affect the system's performance to determine which parameter combinations are ``optimal''~\cite{kang_approach_2011}.
The process contains two phases a search and a design evaluation.
The search find the different combinations of parameters (designs) while the design evaluation evaluates them.

%\section{Related Work}\label{sc:related}
The study of semantics and verification of co-simulation algorithms is presented in \cite{Gomes2019c,Gomes2019a,Broman2013}. 
The paper \cite{Gomes2019c} describes a formalization of an FMI-based co-simulation scenario where several correctness criteria are placed on the co-simulation algorithm to generate and verify a co-simulation algorithm. 
Furthermore, this paper extends their work by treating co-simulation scenarios subject to algebraic loops and adaptive steps.
Thule et al. \cite{Thule_2018} studied how a co-simulation scenario's characteristics can be used to choose the correct simulation strategy for a given co-simulation algorithm. In \cite{thrane2021}, algorithms of complex scenarios are described, but this paper lacks the feature to verify the correctness of algorithms for complex scenarios.

Broman et al. describe in \cite{Broman2013} an approach to achieve deterministic co-simulation results by placing constraints on the co-simulation scenario to avoid algebraic loops. 
They also propose a generic master algorithm for handling step negotiation. 
However, such generic algorithms do not consider other constraints on the SUs like reactive inputs or algebraic loops. This paper deals with all these constraints.

Formal methods have previously been successfully used in the area of co-simulation \cite{Amalio2016,sampaio_behavioural_2016,cerone_formalising_2018}.
Amálio et al.\ \cite{Amalio2016} study how connections between simulation units can be formalized. They investigate how different formal tools can detect algebraic loops to obtain a deterministic co-simulation result. 
Cavalcanti et al.\ \cite{sampaio_behavioural_2016} claim to provide the first behavioral semantics of FMI. The paper shows how to prove essential properties of master algorithms, like termination and determinism. It also shows that the example provided in the FMI standard is not a valid algorithm. The paper \cite{cerone_formalising_2018} by Zeyda et al. formalizes models and proofs about co-simulation in Isabelle/UTP, illustrated by an industrial case study from the railway sector. 
However, their approach does not cover complex scenarios, unlike ours.

Nyman et al. in \cite{jensen_integrating_2017} examine how UPPAAL can analyze controller-based systems with FMUs and a master algorithm modeled in UPPAAL. UPPAAL was used to verify the properties of the controller used in the co-simulation. Palmieri et al. in \cite{palmieri2019framework} have used UPPAAL to provide sound guarantees on the interleaving between a graphical user interface and a generic FMI master algorithm. 
Our approach is more generic and relies on a parameterized template approach applied to arbitrary co-simulation scenarios subject to step negotiation and algebraic loops.

\simon{Add references to synthesizing papers, verification paper}

%
\section{Modeling Co-simulation Scenarios in Maude}
\label{sec:model}

This section describes how we model  individual SUs and their
composition in a co-simulation scenario in Maude.
%
Due to space limitations, we only provide fragments of our Maude
model. 
The entire model, including the synthesis and execution of co-simulation
algorithms (\Cref{sc:synthesize}) and the synthesis of
instrumentations and parameters (\Cref{sc:DSE})  is
available at
\url{https://github.com/SimplisticCode/Co-simulation_WRLA} and
consists of around  
1400 LOC. 

We formalize  co-simulation scenarios in an object-oriented style. The
 state is a term
 \texttt{\char123}$\mathit{SUs}\;\mathit{connections}\;\mathit{orchObjects}$\texttt{\char125}
 of sort \texttt{GlobalState}, 
 where $\mathit{SUs}$ is set of objects modeling simulation units,
$\mathit{connections}$ denote the port couplings, and
$\mathit{orchObjects}$ are two additional objects used
during synthesis and execution of co-simulation algorithms (see
\Cref{sc:synthesize}). 

% We illustrate our  framework using a system where a
% \emph{controller} 
% controls the water level of a \emph{water tank} with constant inflow
% of water,  by opening and closing a valve
% at the bottom of the tank. 
% The system can be modeled using  one SU for the tank and one for
% the controller, and has the architecture shown in
% \cref{fig:simpleexample}.

%\subsection{Simulation Units}
A simulation unit  is modeled as an object instance of the following
class: % a
                                % subsystem with parameters, ports, a
% state, and a time: 
\small
\begin{alltt}
class SU | time\,\,: Nat,                inputs\,\,: Configuration,
           outputs\,\,: Configuration,   canReject\,\,: Bool,
           fmistate\,\,: fmiState,       parameters\,\,: LocalState,
           localState\,\,: LocalState .
\end{alltt}
\normalsize

\noindent The attribute \texttt{time} denotes the time of the SU; 
\texttt{inputs} and \texttt{outputs} denote the objects modeling the
SU's input and output 
ports; \texttt{canReject}  is \texttt{true} if  the SU
implements error estimation (i.e., is an element of the set
$\mayReject$);  
\texttt{fmistate} denotes the \emph{simulation mode}
(see~\cite{FMI2014}) of the SU; \texttt{localState} denotes the SU's internal
state; and \texttt{parameters} denotes the values of the SU's
parameters. 

Input and output ports are modeled as  instances of the following
classes:

\small
\begin{alltt}
class Port | value\,:\,\,FMIValue,\,time\,:\,\,Nat,\,status\,:\,\,PortStatus,\,type\,:\,\,FMIType\,. 
class Input | contract : Contract .
class Output | dependsOn : OidSet .     
subclasses Input Output < Port .
\end{alltt}
\normalsize

\noindent % The attributes
\texttt{value} and \texttt{time} denote, respectively, the
value of the port and the time of its last set/get operation; 
\texttt{status} is \texttt{true}  if the port was
updated  at the current time;   \texttt{contract} denotes the
input port's instrumentation (\texttt{delayed} or
\texttt{reactive}); and 
\texttt{dependsOn} denotes the set of inputs that feed
through to the output port. 

\begin{example}
  We illustrate our  framework using a system where a
\emph{controller} 
controls the water level of a \emph{water tank} with constant inflow
of water,  by opening and closing a valve
at the bottom of the tank. 
The system can be modeled using  one SU for the tank and one SU for
the controller, and has the architecture shown in
\cref{fig:simpleexample}.
%
The water tank  (in its initial state)  is modeled as an object
  
\scriptsize
\begin{alltt}
< "tank" : SU | parameters : ("flow" |-> <\,5\,>),  localState : ("waterlevel" |-> <\,0\,>),
                inputs : (< "valveState" : Input |\,value\,\,:\,\,<\,0\,>, time\,\,:\,\,0,\,contract\,\,:\,\,delayed >),
                outputs : (< "waterlevel" : Output | value : <\,0\,>, time : 0,
                                                     status : Undef, dependsOn : empty >)
                time : 0,  canReject : false >
\end{alltt}
\normalsize

\noindent The \texttt{tank} has one delayed input port and one output
port, and the local state indicates that the tank is empty.  
The parameter \texttt{flow} denotes the amount of water that flows
into the tank per time unit.
\end{example}

To formalize the behaviors of an SU we formalize the operations
\texttt{set}, \texttt{get}, and \texttt{step} in
\cref{def:fmu}. 
For example, the $\fget{}$ operation that updates the \texttt{time}
and \texttt{status}  of a set
of output ports is formalized as follows:\footnote{We do not show
  variable declarations, but follow the convention that variables are
  written with capital letters.} 

\small
\begin{alltt}
  op getAction : Object OidSet -> Object .
  eq getAction(< SU1 : SU | >, empty) = < SU1 : SU | > .
  eq getAction(< SU1 : SU | time : T, 
                            outputs : (< O : Output | > OS) >, (O , P)) = 
     getAction(< SU1 : SU | outputs : 
                     (< O : Output | time : T, status : Def > OS) >, P) .
\end{alltt}
\normalsize

% The operator iterates over a defined set of output ports.
% The \textit{time} and \textit{state} of the output port is update to
% the time $T$ and the input is marked as defined \textit{Def}. 

%The precondition ($\fpreget{}$) is omitted from the operation
%\emph{getAction} since it is included in the functions calling it. 
%The functions $\fget{}$ and $\fset{}$ are similar for all SUs
%described in the framework. 

\noindent The application-specific behavior of an SU is given by
defining its $\fdoStep{}$ 
function:

\begin{example}
The following definition of the \texttt{step} function in our running
example  defines how the  water level of the tank changes as a
function of the 
step duration \texttt{STEP}, the parameter \texttt{flow}, and the
state (\texttt{value}) of the input \texttt{valve}:

\scriptsize
\begin{alltt}
eq step(< "tank" : SU | time : T, parameters : ("flow" |-> <\,FLOW\,>), 
                        inputs : < "valve" : Input | value : <\,STATE\,> >, 
                        outputs : < "waterlevel" : Output | time : T >,
                        localState : ("waterlevel" |-> <\,LEVEL\,>) >,
        STEP) = 
  if STATE == 1 then    \emph{--- valve is open}
    < "tank" : SU | time : (T\,+\,STEP), localState : (\emph{"waterlevel" |-> <\,0\,>}),
          outputs : < "waterlevel" : Output | value\,\,:\,\,<\,0\,>, time\,\,:\,\,(T\,+\,STEP), status\,\,:\,\,Undef > >
  else                  \emph{--- valve is closed}
    < "tank" : SU | time : (T\,+\,STEP),  localState : (\emph{"waterlevel" |-> <\,LEVEL\,+\,(STEP\,*\,FLOW})\,>), 
                    outputs : < "waterlevel" : Output | value : < LEVEL + (STEP * FLOW) >, 
                                                        time : (T + STEP), status : Undef > > 
  fi .
\end{alltt}
\normalsize
\end{example}

%A scenario is a collection of SUs composed using a set of couplings/connections.
A connection/coupling connecting the output port $o$ of SU $\mathit{su}_1$ to
the input port $i$ of SU $\mathit{su}_2$ is represented by the term
$\mathit{su}_1$ \texttt{!} $o$ \texttt{==>}
$\mathit{su}_2$ \texttt{!} $i$.   %  of a subsort
% \texttt{Connection} of sort \texttt{Configuration}.


% ports connects an output with an input:
% \begin{alltt}
%   \small
% op _==>_ : EPortId EPortId -> Connection [ctor] .
% subsort Connection < Configuration .
% \end{alltt}

We define scenarios by defining constants \texttt{simulationUnits}
and \texttt{externalConnection} that denote, resp.,  the
simulation unit objects and  their connections. 

\begin{example}\label{ex:simulationunits}
The SUs and their couplings in our  example are defined as
follows:

\scriptsize
\begin{alltt}
eq simulationUnits = 
   < "tank" : SU | parameters : ("flow" |-> <\,100\,>),  localState : ("waterlevel" |-> <\,0\,>),
                   time : 0,  fmistate : Instantiated, canReject : false, 
                   inputs : (< "valveState" : Input | value : <\,0\,>, type : integer, time : 0,
                                                      contract : delayed, status : Undef >), 
                   outputs : (< "waterlevel" : Output | value : <\,0\,>, type\,:\,integer, time\,:\,0,
                                                        status : Undef, dependsOn : empty >)\,>
   < "ctrl" : SU | parameters : (("high" |-> <\,5\,>) , ("low" |-> <\,0\,>)), canReject : false, 
                   localState : ("valve" |-> <\,false\,>), fmistate : Instantiated, time : 0, 
                   inputs : (< "waterlevel" : Input | value : <\,0\,>, type : integer, time : 0,
                                                      contract : reactive, status : Undef >), 
                   outputs : (< "valveState" : Output | value : <\,0\,>, type\,:\,integer, time\,:\,0,
                                                        status : Undef,\,dependsOn : empty\,>)\,>\,.

eq externalConnection = ("tank" ! "waterlevel" ==> "ctrl" ! "waterlevel") 
                        ("ctrl" ! "valveState" ==> "tank" ! "valveState") .
\end{alltt}
\normalsize
\end{example}

%A scenario is described as a multiset of SUs and connections linking the ports.
%A simulation is instantiated as a \textit{GlobalState} using the equation:
The constant \texttt{setup} defines the initial state, and adds
appropriate initialized orchestration objects to the scenario:

\small
\begin{alltt}
op setup : -> GlobalState .
ceq setup = \char123INIT\char125
  if SCENARIO := externalConnection simulationUnits
  /\char92 validScenario(SCENARIO)
  /\char92 LOOPS := tarjan(SCENARIO)
  /\char92 NeSUIDs := getSUIDsOfScenario(SCENARIO)
  /\char92 INIT := calculateSNSet(SCENARIO OData(1,LOOPS, NeSUIDs)) .
\end{alltt}
\normalsize

\noindent The function \texttt{validScenario} checks whether all inputs are coupled and that no input has two sources.   
The function \texttt{tarjan} returns (a possibly empty) set  of algebraic loops in the scenario by searching for non-trivial strongly connected components in
the graph constructed using the rules  in \cite{Gomes2019c}.   
The function \texttt{getSUIDsOfScenario} returns  the set of all SU
identifiers. Finally, \texttt{calculateSNSet} checks if step negotiation 
should be applied in the simulation of the scenario, and generates a
global initial state with orchestration objects that store information
about the discovered algebraic loops and whether step negotiation is
needed.

%\section{Analysis techniques}\label{sc:casestudy}

The formalization in Maude enables several analyses of both the system of interest and the orchestration algorithm itself. 
This section walks through the different analyses types.

\subsection{Confluent Algorithms}
The described techniques for synthesizing orchestration algorithms will not always lead to one unique orchestration algorithm.
In fact, the rules often permit the synthesizing of multiple distinct orchestration algorithms that all respects the contracts of the scenario.

However, using Maude we can show an important property of the set of allowed - mainly that all of them leads to the same co-simulation result.

This analysis is carried out by Maude's built in LTL model checker using the following command:
$Add LTL expression$
This checks that all possible allowed rewriting of the systems end to same unique final state.

\subsection{Checking Safety properties}
The formalization enables a designer of a CPS to define their own $\fdoStep{}$-functions for the different SUs.  

This allow early and iterative verification of a prototype of the system. 
The perhaps most interesting verification is checking safety properties of the system using Maude's built in LTL-model checker. 

\simon{Should we describe a system here?}

\subsection{Extracting Algorithms}
The synthesized orchestration algorithms 

\subsection{Running co-simulations in Maude}

\begin{itemize}
  \item Show that all synthesized algorithms lead to deterministic co-simulation.
  \item The Water tank example to show how model checking can be useful to do exploratory analysis of the system.
  \item Connecting to the tool to get the correct algorithm for the scenario - external objects.
  \item TODO: Experiment
\end{itemize}


%\section{Concluding Remarks}\label{sc:summary}
This work proposed how rewriting logic can be used to formalize co-simulation semantics such that correct-by-construction co-simulation algorithms can be synthesized and executed.
The approach can handle complex co-simulation scenarios subject to algebraic loops and step rejections.

We demonstrated using model checking that all implementation-aware algorithms of a scenario result in the same co-simulation result.
Furthermore, we presented how a co-simulation practitioner can integrate Maude into the development cycle to explore different system designs and find the optimal parameters and instrumentation of the system.

A future research agenda could into how the formalization could be connected with orchestration engines such that the synthesized algorithms could be executed on real-world simulation units described by the FMI standard.
The parameter search can potentially be explored symbolic.

Future work also includes formalizing the FMI 3.0 standard, which is still under development.
We will use our experience from this work to interact with the FMI steering committee and show them how different semantics affect the simulation result.

\paragraph{Acknowledgements}
We will like to thank Claudio Gomes, Peter Gorm Larsen, Jaco van de Pol, Stefan Hallerstede, and Jose Meseguer for their support and feedback on the work.
%
% ---- Bibliography ---


\bibliographystyle{splncs04}
\bibliography{bib}

\end{document}
