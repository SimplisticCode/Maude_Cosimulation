\subsection{Rewriting Logic and Maude}

Maude~\cite{maude-book} is a rewriting-logic-based executable formal specification language and
high-performance analysis tool for object-based distributed systems.
% for concurrent, object-oriented systems. Maude specifications are
% executable, 
% and the tool provides
% a variety of formal analysis methods,
% including simulation,
% reachability analysis,
% and linear temporal logic (LTL) model checking.

\noindent
A Maude module specifies a 
 \emph{rewrite theory} $(\Sigma, E\cup A,  R)$, where:
\begin{itemize}
\item $\Sigma$ is an algebraic \emph{signature}; i.e., a set of 
\emph{sorts}, \emph{subsorts}, and \emph{function
    symbols}.  
\item $(\Sigma, E\cup A)$ is a \emph{membership equational
logic}
theory,  with  $E$ a set of possibly conditional 
equations and membership axioms,  and
$A$ a set of equational axioms such as 
associativity, commutativity, and identity, so that equational
deduction is performed \emph{modulo} the axioms $A$. 
The theory $(\Sigma, E\cup A)$
specifies
the system's states as members of an algebraic data type.
\item $R$
is a collection of {\em labeled conditional rewrite rules\/} \( [l]:\,
t\longrightarrow t'\,\mbox{ \bf if } \mathit{cond}\), specifying  
the system's local transitions. 
\end{itemize}
A function $f$ is  declared \texttt{op} $f$ : \(s_1\ldots s_n\)
\texttt{->} $s$.
Equations and rewrite rules are introduced with, respectively,
keywords {\tt eq}, or  {\tt ceq}  for 
conditional equations, and 
{\tt rl} and {\tt crl}.
A conditional rewrite rule has the form \texttt{crl\,\,[}$l$\texttt{]\,:}
$t\;$\texttt{=>}$\,\,t'\;$\texttt{if}$\;c_1\,$\texttt{/\char92} \ldots
\texttt{/\char92} $c_n$, where the  conditions $c_1,
\ldots, c_n$ are evaluated from left to right. A condition $c_i$ can be a
Boolean term, an equation, a membership, or a
\emph{matching equation} $u(x_1, \ldots, x_n)\,\,$\texttt{:=}$\,\,u'$
with  variables $x_1, \ldots, x_n$ not appearing in $t$ and not
instantiated in $c_1, \ldots, c_{i-1}$; these variables become
instantiated by \emph{matching} $u(x_1, \ldots, x_n)$ to the normal
form of the (appropriate  instance of) $u'$. $c_i$ can
also be a \emph{rewrite condition} $u_i\,\,$\texttt{=>}$\,\,u_i'$,
which holds if $u_i'$ can be reached in
zero or more rewrite steps from  $u_i$. 
Mathematical variables %in equations and rewrite rules 
 are declared with the keywords {\tt var} and {\tt vars}, 
 or can have the form $var\verb+:+sort$ and be introduced on the fly.
 %Comments are preceded by `\texttt{----}'.
% An equation $f(t_1, \ldots, t_n) = t$ 
% with the \texttt{owise} (``otherwise'') attribute  can be applied to a subterm 
% $f(\ldots)$ only if no other equation with left-hand side
% $f(u_1, \ldots, u_n)$ can be applied. Maude also provides standard
% parameterized data types (sets, maps, etc.) that can be
% instantiated (and  renamed);  for example,
% {\small \texttt{pr SET\char123Nat\char125{} * (sort Set\char123Nat\char125{} to
%   Nats)}} defines  a sort {\small \texttt{Nats}} of \emph{sets} of natural numbers.



A \emph{class} declaration \texttt{class \(C\) |
  \(\mathit{att}_1\)\,\,:\,\,\(\mathit{s}_1\), \dots ,
  \(\mathit{att}_n\)\,\,:\,\,\(\mathit{s}_n\)\,\,} 
declares a class $C$ of objects with attributes $att_1$ to $att_n$ of
sorts $s_1$ 
to $s_n$. An {\em object instance\/} of class $C$  is
represented as a term
$\texttt{<}\: O : C \mid \mathit{att}_1: \mathit{val}_1, \dots , \mathit{att}_n: \mathit{val}_n\:\texttt{>}$,
where $O$, of sort \texttt{Oid},  is the
object's
\emph{identifier}, and where $val_1$ to 
$val_n$ are the current values of the attributes $att_1$ to
$att_n$.
% A \emph{message} is a term of sort \texttt{Msg}. 
A system state
 is modeled as a term of 
the sort \texttt{Configuration}, and   has 
the structure of a  \emph{multiset} made up of objects and messages
(and \emph{connections} in our case). 

The dynamic behavior of a
 system is axiomatized by specifying each of its 
transition patterns by a rewrite rule. For example, 
  the rule (with label \texttt{l})

\small
\begin{alltt}
  rl [l] :  < O : C | a1 : f(x, y), a2 : O', a3 : z >
       =>   < O : C | a1 : x + z,   a2 : O', a3 : z > .
\end{alltt}
\normalsize

\noindent  defines a family of transitions 
in which the attribute \texttt{a1} of object \texttt{O} is updated to \texttt{x\,+\,z}. 
Attributes whose values do not change and do not affect
the next state,  
 such as \texttt{a2} and the right-hand side occurrence of \texttt{a3}, need not be mentioned. % Note that in the above rule \texttt{O}, declared as either
       % constant or variable of sort \texttt{Oid}, is the object's
       % identifier that can also be a parameter of some function
       % (e.g., the message function \texttt{m}). 
  %
% Maude also supports \emph{metaprogramming}
% in the sense that a Maude specification $\mathit{M}$  
% can be represented as a \emph{term} $\overline{M}$ (of sort
% \texttt{Module}), so that a module transformation can be defined as a
% Maude function
% $f: \mathtt{Module} \rightarrow \mathtt{Module}$. 

 \paragraph{Formal Analysis in Maude.} 
Maude provides a number of  analysis methods,
including rewriting for simulation purposes, reachability analysis,
and linear temporal logic (LTL) model checking.
The rewrite command \texttt{frew} $\mathit{init}$ simulates one
behavior from the initial state/term $\mathit{init}$ by applying
rewrite rules.  Given an initial state $\mathit{init}$, a state pattern
 $\mathit{pattern}$ and an (optional) condition $\mathit{cond}$,
 Maude's  \texttt{search}
 command searches the  reachable state space from $\mathit{init}$ for
 all (or optionally a given number of) states that match
 $\mathit{pattern}$ such that 
 $\mathit{cond}$ holds:   
 
\small
\begin{alltt}
  search \(\mathit{init}\)  =>!  \(\mathit{pattern}\) such that \(\mathit{cond}\) .
\end{alltt}
\normalsize 

\noindent The arrow  \texttt{=>!} means that Maude only searches for 
 \emph{final} states (i.e., states that cannot be further rewritten)
 that match $\mathit{pattern}$ and satisfies $\mathit{cond}$. If the
 arrow  is  \texttt{=>*} then Maude searches for all
 reachable states satisfying the search condition. 

% \paragraph{Metaprogramming.} Maude supports metaprogramming
% in the sense that a Maude specification $\mathit{M}$  
% can be represented as a \emph{term} $\overline{M}$ (of sort
% \texttt{Module}), so that a module transformation can be defined as a
% Maude function
% $f: \mathtt{Module} \rightarrow \mathtt{Module}$. 
% in Maude. We can write a Maude metaprogram by importing the
% \texttt{META-LEVEL} module into another module defining a function
% $f$, which has $\overline{M}$ as its argument, and transforms it to
% another Maude specification $\mathit{f(\overline{M})}$. In this paper
% we use such a transformation technique for rewrite rule
% instrumentation. Specifically, a user-provided Maude model of rewrite
% rules is transformed to the one with monitoring objects attached,
% which gather data from the original model for analysis purpose. The
% transformation is transparent to the user, meaning that it should not
% modify the original model's state or behavior. 


%A metaprogram in Maude can take programs/modules as inputs and
%performs some useful computation such as transforming one Maude
%program/module into another. 