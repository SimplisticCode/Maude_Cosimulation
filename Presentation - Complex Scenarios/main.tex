\documentclass{beamer}
\usepackage[utf8]{inputenc}

\usepackage{fourier} %font utopia imported
\usetheme{Madrid}
\usecolortheme{default}

%------------------------------------------------------------
%This block of code defines the information to appear in the
%Title page
\title[Synthesizing Co-Simulation Algorithms] %optional
{Synthesizing and Verifying Orchestration Algorithms}
\date{\today}
\usepackage{booktabs}
\usepackage{natbib}
\usepackage{multicol}
\usepackage{listings}
\usepackage{environ}
\usepackage{graphicx}
\usepackage{amsfonts}
\usepackage{amsmath}
\usepackage{amssymb}
\usepackage{stmaryrd}
\usepackage{amstext}
\usepackage{semantic}
\usepackage{semantic}
\usepackage{algorithm}
\usepackage{algpseudocode}
\usepackage{etoolbox}
\usepackage{paralist}
\usepackage{verbatim}


\usepackage[capitalise,nameinlink]{cleveref}

% Comments

\newcommand{\simon}[1]{%
    {\scriptsize
        \textbf{\textcolor{red}{Simon: #1}}
    }%
}%

\newcommand{\jaco}[1]{%
    {\scriptsize
        \textbf{\textcolor{blue}{Jaco: #1}}
    }%
}%

\newcommand{\claudio}[1]{%
    {\scriptsize
        \textbf{\textcolor{magenta}{Claudio: #1}}
    }%
}%



% Paper specific notation

\setlength{\intextsep}{10pt}

\crefname{problem}{problem}{problems}

\newcommand{\inputV}{v}
\newcommand{\consistent}{\ensuremath{\mathit{Consistent}}}
\newcommand{\remaining}{\ensuremath{\mathit{Remaining}}}
\newcommand{\dontcare}{\_}
\newcommand{\defined}{\ensuremath{\mathit{defined}}}
\newcommand{\undefined}{\ensuremath{\mathit{undefined}}}
\newcommand{\properties}{P}
\newcommand{\satisfies}{\vDash}
\newcommand{\simulator}{\mathcal{A}}
\newcommand{\Induced}[2]{\llbracket #1 \rrbracket_{#2}}
\newcommand{\timebase}{\setreal_{\geq 0}}
\newcommand{\stepbase}{\setreal_{> 0}}
\newcommand{\backtrack}{B}

\newcommand{\stateset}[1]{S_{#1}}
\newcommand{\runstate}[1]{S^{R}_{#1}}
\newcommand{\state}[1]{s_{#1}}
\newcommand{\inputs}[1]{U_{#1}}
\newcommand{\inputvar}[1]{u_{#1}}
\newcommand{\outputs}[1]{Y_{#1}}
\newcommand{\outputvar}[1]{y_{#1}}
\newcommand{\mayReject}{\mathcal{M}}
\newcommand{\values}{\mathcal{V}}
\newcommand{\true}{\mathit{true}}
\newcommand{\false}{\mathit{false}}
\newcommand{\feedthrough}[1]{D_{#1}}
\newcommand{\reactivity}[1]{R_{#1}}
\newcommand{\fset}[1]{\mathtt{set}_{#1}}
\newcommand{\fget}[1]{\mathtt{get}_{#1}}
\newcommand{\fdoStep}[1]{\mathtt{doStep}_{#1}}
\newcommand{\fSave}[1]{\mathtt{Save}_{#1}}
\newcommand{\fRestore}[1]{\mathtt{Restore}_{#1}}
\newcommand{\maxStep}[1]{\mathtt{h}_{#1}}
\newcommand{\timestamp}[1]{\varphi(#1)}
\newcommand{\feedsto}[2]{U_{#1}^{#2}}
\newcommand{\CheckCon}{\mathit{CheckCon}}
\newcommand{\stepfound}{\mathit{Step\_found}}

\newcommand{\complexity}{Com}
\newcommand{\master}{\mathcal{A}}
\newcommand{\alloutputs}{Y}
\newcommand{\allfeedthroughs}{D}
\newcommand{\allreactivity}{R}
\newcommand{\allcontracts}{\mathcal{C}}
\newcommand{\coupling}{L}
\newcommand{\allinputs}{U}
\newcommand{\fmus}{C}
\newcommand{\fmusLoop}{C_{loop}}
\newcommand{\sequence}[1]{\pargroup{#1}}
\newcommand{\functioncall}{f}
\newcommand{\initcall}{I}
\newcommand{\allfunctioncalls}{F}
\newcommand{\fmu}[1]{\texttt{#1}}
\newcommand{\signal}[1]{\texttt{#1}}
\newcommand{\before}[2]{\ensuremath{#1 \twoheadrightarrow #2}}
\newcommand{\ibefore}[2]{\ensuremath{#1 \rightarrow #2}}
\newcommand{\after}[1]{{#1}'}
\newcommand{\aftern}[2]{{#1}^{(#2)}}
\newcommand{\stateafter}[2]{\ensuremath{\state{#1}^{(#2)}}}
\newcommand{\minstep}{\ensuremath{h_{min}}}

\algnewcommand\algorithmicforeach{\textbf{for each}}
\algdef{S}[FOR]{ForEach}[1]{\algorithmicforeach\ #1\ \algorithmicdo}
% New definitions
\algnewcommand\algorithmicswitch{\textbf{switch}}
\algnewcommand\algorithmiccase{\textbf{case}}
\algnewcommand\algorithmicassert{\texttt{assert}}
\algnewcommand\Assert[1]{\State \algorithmicassert(#1)}%
% New "environments"
\algdef{SE}[SWITCH]{Switch}{EndSwitch}[1]{\algorithmicswitch\ #1\ \algorithmicdo}{\algorithmicend\ \algorithmicswitch}%
\algdef{SE}[CASE]{Case}{EndCase}[1]{\algorithmiccase\ #1}{\algorithmicend\ \algorithmiccase}%
\algtext*{EndSwitch}%
\algtext*{EndCase}%






% Generic stuff

\newcommand{\footurl}[1]{\footnote{\url{#1}}}

% Math
\newcommand{\brackets}[1]{\ensuremath{ \left[ #1 \right] }}
\newcommand{\tuple}[1]{\ensuremath{ \left\langle #1 \right\rangle }}
\newcommand{\set}[1]{\ensuremath{ \left\{ #1 \right\}}}
\newcommand{\system}[1]{\ensuremath{ \begin{cases} #1 \end{cases}}}
\newcommand{\rightgroup}[1]{\ensuremath{ \left. \begin{matrix} #1 \end{matrix} \right\} } }
\newcommand{\pargroup}[1]{\ensuremath{ \left( #1 \right)}}
\newcommand{\inv}[1]{\ensuremath{\pargroup{ #1 }^{-1}}}
\newcommand{\dert}[1]{\ensuremath{ \dot{#1} }}
\newcommand{\ddert}[1]{\ensuremath{ \ddot{#1} }}
\newcommand{\partialder}[2]{\ensuremath{ \frac{\partial#1}{\partial#2} }}
\newcommand{\setreal}[0]{\ensuremath{\mathbb{R}}}
%\newcommand{\setbool}[0]{\ensuremath{\mathit{Bool}}}
\newcommand{\setnat}[0]{\ensuremath{\mathbb{N}}}
\newcommand{\norm}[1]{\left\lVert#1\right\rVert}
\newcommand{\bnorm}[1]{\big\lVert#1\big\rVert}
\newcommand{\abs}[1]{\left|#1\right\|}
\newcommand{\xs}[2]{\ensuremath{#1^{\left[#2\right]}}}
\newcommand{\infinitynorm}[1]{\left\lVert#1\right\rVert_\infty}


\newcommand{\vectorOne}[1]{\brackets{%
\begin{matrix}
  #1
 \end{matrix}%
}}
\newcommand{\vectorTwo}[2]{\brackets{%
\begin{matrix}
  #1 \\
  #2
 \end{matrix}%
}}
\newcommand{\vectorThree}[3]{\brackets{%
\begin{matrix}
  #1 \\
  #2 \\
  #3
 \end{matrix}%
}}
\newcommand{\vectorFour}[4]{\brackets{%
\begin{matrix}
  #1 \\
  #2 \\
  #3 \\
  #4
 \end{matrix}%
}}
\newcommand{\vectorFive}[5]{\brackets{%
\begin{matrix}
  #1 \\
  #2 \\
  #3 \\
  #4 \\
  #5
 \end{matrix}%
}}
\newcommand{\vectorSix}[6]{\brackets{%
\begin{matrix}
  #1 \\
  #2 \\
  #3 \\
  #4 \\
  #5 \\
  #6
 \end{matrix}%
}}
\newcommand{\vectorSeven}[7]{\brackets{%
\begin{matrix}
  #1 \\
  #2 \\
  #3 \\
  #4 \\
  #5 \\
  #6 \\
  #7
 \end{matrix}%
}}
\newcommand{\vectorEight}[8]{\brackets{%
\begin{matrix}
  #1 \\
  #2 \\
  #3 \\
  #4 \\
  #5 \\
  #6 \\
  #7 \\
  #8
 \end{matrix}%
}}

\newenvironment{aligneq*}%
{
\begin{equation*}
\begin{aligned}
}{
\end{aligned}
\end{equation*}
}

\newenvironment{aligneq}%
{
\begin{equation}
\begin{aligned}
}{
\end{aligned}
\end{equation}
}



%enable \cref{...} and \Cref{...} instead of \ref: Type of reference included in the link
%Nice formats for \cref
\usepackage{iflang}
\IfLanguageName{ngerman}{
  \crefname{table}{Tab.}{Tab.}
  \Crefname{table}{Tabelle}{Tabellen}
  \crefname{figure}{\figurename}{\figurename}
  \Crefname{figure}{Abbildungen}{Abbildungen}
  \crefname{equation}{Gleichung}{Gleichungen}
  \Crefname{equation}{Gleichung}{Gleichungen}
  \crefname{listing}{\lstlistingname}{\lstlistingname}
  \Crefname{listing}{Listing}{Listings}
  \crefname{section}{Abschnitt}{Abschnitte}
  \Crefname{section}{Abschnitt}{Abschnitte}
  \crefname{paragraph}{Abschnitt}{Abschnitte}
  \Crefname{paragraph}{Abschnitt}{Abschnitte}
  \crefname{subparagraph}{Abschnitt}{Abschnitte}
  \Crefname{subparagraph}{Abschnitt}{Abschnitte}
}{
  \crefname{section}{Sect.}{Sect.}
  \Crefname{section}{Section}{Sections}
  \crefname{listing}{\lstlistingname}{\lstlistingname}
  \Crefname{listing}{Listing}{Listings}
}
\crefname{section}{Sect.}{Sect.}
\Crefname{section}{Section}{Sections}
\crefname{listing}{\lstlistingname}{\lstlistingname}
\Crefname{listing}{Listing}{Listings}

\crefname{assumption}{Assumption}{Assumptions}




\urlstyle{same}
\author[Thrane, S.] % (optional)
{Simon Thrane Hansen \\ Ph.D. Student AU}


\date[AU 2021] % (optional)
{\today}

%\logo{\includegraphics[height=1.5cm]{lion-logo.jpg}}

%End of title page configuration block
%------------------------------------------------------------



%------------------------------------------------------------
%The next block of commands puts the table of contents at the 
%beginning of each section and highlights the current section:


%------------------------------------------------------------
\begin{document}

%The next statement creates the title page.
\frame{\titlepage}

\begin{frame}{Overview}
    \tableofcontents
\end{frame}

\section{Recap Simple Scenarios}
\begin{frame}{Simulation of simple scenario}
    Last time we covered simple scenarios by describing the rules of the actions that formed the step operation graph - that for simple scenarios only contain trivial SCC.
    \begin{columns}[T] % align columns
        \begin{column}{.48\textwidth}
            \begin{figure}    
                \includegraphics[width=0.5\textwidth]{images/simple_example.pdf}
            \end{figure}
            \begin{figure}
                \centering
                \includegraphics[width=0.55\textwidth]{images/simple_scenario_graph.pdf}
            \end{figure}
    \end{column}%
    \hfill%
    \begin{column}{.48\textwidth}
        \begin{algorithm}[H]
            \caption{Step function of scenario}
            \begin{algorithmic}[1]
              \scriptsize
              \State $\stateafter{A}{s+H} \gets \fdoStep{A}(\stateafter{A}{s}, H)$
              \State $\stateafter{B}{s+H} \gets \fdoStep{B}(\stateafter{B}{s}, H)$
              \State $F_v \gets \fget{B}(\stateafter{B}{s+H}, \outputvar{F})$
              \State $\stateafter{A}{s+H} \gets \fset{A}(\stateafter{a}{s+H}, \inputvar{F}, F_v)$
              \State $x_v \gets \fget{A}(\stateafter{A}{s+H}, \outputvar{x})$
              \State $\stateafter{B}{s} \gets \fset{B}(\stateafter{B}{s+H}, \inputvar{x}, x_v)$
            \end{algorithmic}
          \end{algorithm}
    \end{column}%
    \end{columns}
\end{frame}


\section{Complex Scenarios}

\begin{frame}{Introduction to complex scenarios}
    
\end{frame}


\begin{frame}{Algebraic Loops}
    \textbf{Happens if}: The value on a ports depends on itself.\\
    \textbf{Example of the two kinds of Algebraic Loops}: 
    \begin{columns}[T] % align columns
        \begin{column}{.48\textwidth}
            \begin{figure}    
                \includegraphics[width=0.9\textwidth]{images/reactive_loop_scenario.pdf}
                \caption{Reactivity Loop}
            \end{figure}
    \end{column}%
    \hfill%
    \begin{column}{.48\textwidth}
        \begin{figure}    
            \includegraphics[width=0.9\textwidth]{images/feedthroughloop.pdf}
            \caption{Feed-through Loop}
        \end{figure}
    \end{column}%
    \end{columns}
    \textbf{How do we simulate this}: Fixed-point iteration - The simulation iteratively searches for a fixed-point on all the output ports between two successive iterations.\\
    \textbf{Problem:} Introduces complex SCCs in the step operation graph - no topological order can be found.
\end{frame}

\begin{frame}{Step negotiation}
    \textbf{Happens if}: A scenario contains SUs that may reject to take a step of arbitrary size.\\
    \textbf{Example}: 
    \begin{columns}[T] % align columns
        \begin{column}{.48\textwidth}
            \begin{figure}    
                \includegraphics[width=0.9\textwidth]{images/step_scenario.pdf}
            \end{figure}
    \end{column}%
    \hfill%
    \begin{column}{.48\textwidth}
        \begin{figure}    
            \includegraphics[width=0.7\textwidth]{images/step_scenario_original.pdf}
        \end{figure}
    \end{column}%
    \end{columns}
    \textbf{How do we simulate this}: Step finding procedure - The simulation iteratively searches for a step all SUs are able to perform.\\
    \textbf{Problem with the traditional method:} Fails to identify these scenarios and can therefore not treat them.
\end{frame}

\begin{frame}{Simulation of complex scenario}
    \begin{columns}[T]
        \begin{column}{0.5\textwidth}
            \begin{itemize}
                \item Should be simulated using a correct configuration (Step size + Fixed-points).
                \item An incorrect configuration breaks the preconditions.
                \item The simulation strategy is to use an iterative search for a correct (stable) configuration.
                \item The simulation is restarted after unsuccessful tries (using restore and save).
            \end{itemize}
        \end{column}
        \begin{column}{0.5\textwidth}
            \begin{algorithm}[H]
                \caption{Simulation of complex scenarios.}
            \label{alg:algorithm_step}
            \begin{algorithmic}[1]
              \scriptsize
                \State Save SUs
                \While{$!correctConfiguration$} 
                \State Iterative search
                \If{$!correctConfiguration$}
                    \State Restore SUs
                \EndIf
                \EndWhile
            \end{algorithmic} 
          \end{algorithm}
        \end{column}
    \end{columns}    
\end{frame}


\begin{frame}{Challenges in Synthesizing Algorithms for Complex Scenarios}
    \begin{enumerate}
        \item Detection of complex scenarios
        \item Extracting the Algorithm
        \begin{itemize}
            \item The graph contains cycles
        \end{itemize}
    \end{enumerate}
    \begin{itemize}
        \item Complex Scenarios - having SCC in the graph.
        \begin{itemize}
            \item Solving Algebraic Loops (cycles in the graph)
            \item Performing Step Negotiation to ensure that SUs move in lockstep.
        \end{itemize}
    \end{itemize}
    These are solved using specialize heuristics to break the SCCs.
    \begin{columns}[T] 
        \begin{column}{.48\textwidth}
            \begin{figure}    
                \includegraphics[width=0.9\textwidth]{images/reactive_step_graph.pdf}
            \end{figure}  
        \end{column}
    \hfill%
    \begin{column}{.48\textwidth}
        \begin{figure}    
            \centering
            \includegraphics[width=0.9\textwidth]{images/step_scenario_org.pdf}
        \end{figure}
    \end{column}
    \end{columns}
\end{frame}

\begin{frame}{Synthesizing Complex Scenarios - Algebraic Loop}
    The SCC is broken using reduction scheme:
    \begin{columns}[T] 
        \begin{column}{.48\textwidth}
            \begin{figure}    
                \includegraphics[width=0.8\textwidth]{images/jacobian_reduced_graph.pdf}
                \caption{Maximal reduction.}
            \end{figure}  
        \end{column}
    \hfill%
    \begin{column}{.48\textwidth}
        \begin{figure}    
            \centering
            \includegraphics[width=0.8\textwidth]{images/gauss_reduced_graph.pdf}
            \caption{Minimal reduction.}
        \end{figure}
    \end{column}
    \end{columns}
\end{frame}


\begin{frame}{Synthesizing Complex Scenarios - Step negotiation}
    \begin{columns}[T] 
        \begin{column}{.48\textwidth}
            \begin{figure}    
                \includegraphics[width=0.8\textwidth]{images/step_scenario_graph.pdf}
                \caption{Step graph - the blue part shows the iterative search.}
            \end{figure}  
        \end{column}
    \hfill%
    \begin{column}{.48\textwidth}
        \begin{algorithm}[H]
            \caption{Step negotiation.}
        \begin{algorithmic}[1]
          \scriptsize
            \While{$!\stepfound$} 
            \State $(\stateafter{D}{s+h_D},h_D) \gets \fdoStep{D}(\stateafter{D}{s}, h)$
            \State $g_v \gets \fget{D}(\stateafter{D}{s+h_D}, \outputvar{g})$       
            \State $\stateafter{C}{s} \gets \fset{C}(\stateafter{C}{s}, \inputvar{G}, G_v)$
            \State $(\stateafter{C}{s+h_C},h_C) \gets \fdoStep{C}(\stateafter{C}{s}, h_D)$
            \State $h \gets min(h_C, h_D)$
            \State $\stepfound \gets h == h_C \land h == h_D$
            \If{$!\stepfound$}
                \State Restore SUs
            \EndIf
            \EndWhile
        \end{algorithmic} 
      \end{algorithm}
    \end{column}
    \end{columns}
\end{frame}


\end{document}